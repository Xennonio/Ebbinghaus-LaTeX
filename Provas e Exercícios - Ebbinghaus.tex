\documentclass[11pt]{article}

%% Packages
\usepackage{amsmath,amsthm,amsfonts,amssymb,amscd}
\usepackage{multirow,booktabs}
\usepackage[table]{xcolor}
\usepackage{fullpage}
\usepackage{lastpage}
\usepackage{enumitem}
\usepackage{fancyhdr}
\usepackage{mathrsfs}
\usepackage{wrapfig}
\usepackage{setspace}
\usepackage{calc}
\usepackage{multicol}
\usepackage{cancel}
\usepackage[retainorgcmds]{IEEEtrantools}
\usepackage[margin=3cm]{geometry}
\usepackage{amsmath}
\usepackage{empheq}
\usepackage{framed}
\usepackage[most]{tcolorbox}
\usepackage{xcolor}
\usepackage{proof}
\usepackage{mathabx}

%% Pagestyle
\newlength{\tabcont}
\setlength{\parindent}{0.0in}
\setlength{\parskip}{0.05in}
\colorlet{shadecolor}{orange!15}
\parindent 0in
\parskip 12pt
\geometry{margin=1in, headsep=0.25in}
\theoremstyle{definition}
\newtheorem{defn}{Definição}
\newtheorem{exer}{Exercício}
\newtheorem{note}{Nota}
\newtheorem{theorem}{Teorema}
\newtheorem{corollary}{Corolário}
\newtheorem{lemma}{Lema}

%% NewCommands
\newcommand{\sse}{\leftrightarrow}
\newcommand{\mc}[1]{\mathcal{#1}}
\newcommand{\mf}[1]{\mathfrak{#1}}
\newcommand{\msf}[1]{\mathsf{#1}}
\newcommand{\mbb}[1]{\mathbb{#1}}
\newcommand{\ol}[1]{\overline{#1}}
\newcommand\overtext[2]{\stackrel{\mathclap{\normalfont\mbox{#1}}}{#2}}

%% Document

\begin{document}
\setcounter{section}{0}
\thispagestyle{empty}

\begin{center}
{\LARGE \bf Provas e Exercícios}\\
{\large Ref. H. D. Ebbnghaus}\\
Primavera 2022
\end{center}

\tableofcontents

\section{Sintaxe das Linguagens de Primeira Ordem}

\textbf{Exercício 1.3.}
Seja $\alpha:\mbb{N}\to\mbb{R}$ dado. Para $a,b\in\mbb{R}$ tq $a<b$ mostre que $\exists c\in I:=[a,b]$ tq $c\notin\text{Im}(\alpha)$. Conclua disso que $I$, e portanto $\mbb{R}$, são incontáveis.
\begin{proof}
    Seja $I_0:=[a,b]$ e defina indutivamente $I_{n+1}:=I_n\backslash\{\alpha(n)\}$, obviamente $I_0 \supseteq I_1 \supseteq \dots$ forma uma sequência de intervalos encaixantes e, por estarmos em $\mbb{R}$, vale que
    $$\bigcap_{n\in\mbb{N}}I_n\ne\emptyset$$
    como $\alpha(k)\notin  I_n,\forall k\le n$, então, em particular, $\alpha(k)\notin\bigcap_{n\in\mbb{N}}I_n, \forall k\in\mbb{N}$, i.e., $c\notin\bigcap_{n\in\mbb{N}}\forall c\in\text{Im}(\alpha)$
\end{proof}

\hrule

\textbf{Exercício 1.4.}
Prove que se $M_0,M_1,\dots\preceq\aleph_0$, então
\[
\bigcup_{n\in\mbb{N}}M_n\preceq\aleph_0
\]
e o utilize para provar o \textbf{Lema 1.2}.
\begin{proof}
    O primeiro teorema é facilmente provável utilizando um argumento similar ao da Diagonal de Cantor: Sabemos que existe uma bijeção $\alpha_i:\mbb{N}\to M_i,\forall i\in\mbb{N}$, equivalente a uma sequência $\alpha_{i0},\alpha_{i1},\dots$ tq $M_i=\{\alpha_{in}\mid n\in\mbb{N}\}$, construindo a matriz $a_{ij}:=\alpha_{ij}$ utilizamos a diagonal para enumerar todos os elementos da matriz.\\\\
    Definindo $M_n:=\mbb{A}^n$, o conjunto de strings de comprimento $n$, é fácil mostrar por indução que $M_i\preceq\aleph_0, \forall i\in\mbb{N}$, com isso
    \[
    \bigcup_{n\in\mbb{N}}M_n=\mbb{A}^*\preceq\aleph_0.
    \]
    Como $\mbb{A}^*\succeq\aleph_0$ pelo teorema de Schröder-Bernstein $\mc{A}^*\approx\aleph_0$.
\end{proof}

\hrule

\textbf{Exercício 1.5.} Demonstre o Teorema de Cantor: não existe $\alpha:M\to\mc{P}(M)$ sobrejetivo e, portanto, bijetivo.
\begin{proof}
    Seja $S:=\{a\in M\mid a \notin\alpha(a)\}\in\mc{P}(M)$, assumindo por hipótese que existe $\alpha$ sobrejetivo, então $\exists s\in M$ tq $\alpha(s)=S$. Se $s\in S$, por definição $s\notin\alpha(s)=S$, contradição. Se $s\notin S$, por definição $s\in\alpha(s)=S$, contradição, portanto não existe tal $s\in M$.
\end{proof}

\hrule

\textbf{Exercício 4.6.}
(a) Seja $\mf{C}_v$ o cálculo consistindo das seguintes regras:
\[
\infer[;]{x~~~x}{\phantom{aaaaaaa}} ~~~~\infer[\text{ se }f\in\mc{S}\text{ é n-ária e }i\in\{1,\dots,n\}.]{y~~~ft_1\dots t_n}{y & t_i}
\]
Mostre que para toda variável $x$ e $\mc{S}$-termo $t$, $x$ $t$ é derivável em $\mf{C}_v$ sse $x\in\msf{var}(t)$.\\
(b) Dê um resultado para $\msf{SF}$ análogo ao resultado para $\msf{var}$ em (a).

\begin{proof}
    (a)\\
    (i) Se $x\in\msf{var}(t)$ então $x$ $t$ é derivável em $\mf{C}_v$: Se $t=x$ então $x\in\msf{var}(t)$ e pela 1ª regra $x$ $t$ é derivável. Se $t=t_i$ e $x\in\msf{var}(t_i)$ então, seguindo a definição, $x\in f(t_1\dots t_n)$.\\
    (ii) Se $x$ $t$ é derivável em $\mf{C}_v$ então $x\in\msf{var}(t)$: Se $t=x$ a primeira regra garante que $x\in\msf{var}(t)$. Se $t=ft_1\dots t_n$ então existe um $x$ $t_i$ em $\mf{C}_v$, como todos termos dessa forma que existem partem de uma regra sem premissa (regra 1) então $x\in\msf{var}(t_i)$ logo $x\in\msf{var}(ft_1\dots t_n)$.\\
    (b) Seja o cálculo $\mf{C}_a$ definido pelas regras:
    \[
    \infer[;]{t_m\overtext{.}{=} t_n~~~t_m\overtext{.}{=} t_n}{\phantom{aaaaaaa}} ~~~~\infer[;]{\varphi~~~\neg\psi}{\varphi & \psi}~~~~\infer[*=\wedge,\vee,\to,\sse;]{\varphi~~~(\varphi*\psi)}{\varphi & \psi}~~~~\infer[Q=\forall,\exists.]{\varphi~~~Qx\psi}{\varphi & \psi}
    \]
    Para todo termo $t_m,t_n$ e toda variável $x$. $\varphi$ $\psi$ é derivável em $\mf{C}_a$ sse $\varphi\in\msf{SF}(\psi)$.
\end{proof}

\hrule

\textbf{Exercício 4.7.} Altere o cálculo de fórmulas omitindo os parênteses que delimitam as fórmulas introduzidas da forma $\varphi\square\psi$. Mostre que tais fórmulas não terão mais uma única decomposição e que $\msf{SF}$ não será mais uma função bem definida.

\begin{proof}
    Pegue, por exemplo, a fórmula $\varphi:=\exists xPx\wedge Qy$, podemos, utilizando o cálculo de fórmulas, construir duas derivações diferentes da mesma fórmula:
    \begin{enumerate}
        \item $Px$, (F2) em $P$ e $x$;
        \item $Qy$, (F2) em $Q$ e $y$;
        \item $Px\wedge Qy$, (F4) em (1) e (2) com $\wedge$;
        \item $\exists xPx\wedge Qy$, (F5) em (3) usando $\exists$ e $x$.
    \end{enumerate}
    e a outra altera somente os passos (3) e (4) para:
    \begin{enumerate}
        \item $\exists xPx$, (F5) em (1) usando $\exists$ e $x$;
        \item $\exists xPx\wedge Qy$ $x$, (F4) em (2) e (3) com $\wedge$.
    \end{enumerate}
    Obviamente $\msf{SF}(\varphi)=\{\varphi,Px\wedge Qy,Qy,Px\}$ utilizando a primeira derivação e $\msf{SF}(\varphi)=\{\varphi, \exists xPx, Qy, Px\}$ utilizando a segunda.
\end{proof}

\hrule

\textbf{Exercício 4.8.} Definimos uma $\mc{S}$-fórmula em notação polonesa ($\mc{S}$-$P$-fórmula) como as strings em $\mbb{A}_{\mc{S}}$ tq a regra (F4) é alterada pra: Se $\varphi,\psi$ são $\mc{S}$-$P$-fórmulas, então também são $\square\varphi\psi$, com $\square=\wedge,\vee,\to,\leftrightarrow$.

\begin{proof}
    Precisamos antes provar o análogo ao \textbf{Lema 4.2.(b)} para $\mc{S}$-$P$-fórmulas: para $\varphi\ne\varphi'$, $\varphi$ não é um segmento inicial próprio de $\varphi'$. Se $\varphi=\wedge\chi\psi$, assuma por contradição que $\varphi$ é um segmento inicial próprio de $\varphi'$, i.e., existe $\zeta\ne\square$ tq $\varphi\zeta=\wedge\psi\chi=\varphi'$, mas como $\varphi'$ começa com $\wedge$ este só pode ser formado a partir de (F4), portanto $\varphi=\wedge\chi'\psi'$ para algumas $\chi',\psi'$ $\mc{S}$-$P$-fórmulas. Podemos então cancelar $\wedge$ e ficar com $\chi\psi\zeta=\chi'\psi$, mas, pela hipótese de indução, se $\chi$ é um segmento próprio de $\chi'$, só pode ser o caso que $\chi=\chi'$, o mesmo vale para $\psi$ e $\psi'$, logo $\zeta=\square$, contradição.
    
    Para provar o \textbf{Lema 4.3.(b)} provemos primeiro que se $\varphi_1\dots\varphi_n=\varphi_1'\dots\varphi_n'$, então $\varphi_i=\varphi_i'$ por indução. Caso base: $\varphi_1$ é segmento inicial próprio de $\varphi_1'$, pelo \textbf{Lema 4.2.(b)} temos $\varphi_1=\varphi_1'$. Hipótese de indução: assuma que $\varphi_i=\varphi_i'$, logo podemos cancelá-lo, o que implica que $\varphi_{i+1}$ é segmento inicial próprio de $\varphi_{i+1}'$, i.e., $\varphi_{i+1}=\varphi_{i+1}'$.\\
    Seja agora $n\ne m$, assuma sem perda de generalidade que $n = m + k$ para $k > 0$, logo $\varphi_1\dots\varphi_n=\varphi_1'\dots\varphi_n'\dots\varphi_m'$, da prova anterior temos então que $\square=\varphi_{n+1}'\dots\varphi_m'$, contradição, logo $k = 0$.\\
    A prova do \textbf{Lema 4.4.(b)} é trivial, basta definirmos a função $\msf{SF}$ para $\mc{S}$-$P$-fórmulas, o que é muito simples.
\end{proof}

\hrule

\textbf{Exercício 4.9.} Seja $t_1,\dots,t_n\in\mc{T}^\mc{S}$ de comprimento $k$, para $n\ge1$. Mostre que $\exists\xi,\eta\in\mbb{A}^*_\mc{S}$ unicamente determinados e $t\in\mc{T}^\mc{S}$ tq o comprimento de $\xi$ é $1\le i<k$ e $t_1\dots t_n = \xi t\eta$.

\begin{proof}
    Seja $i=\sum_{j=0}^m\msf{lng}(t_j)$ para algum $m<n$, nesse caso $t=t_{m+1}$ e $\eta=t_{m+2}\dots t_{n}$ podendo ser possivelmente $\square$. Se todos os termos são constantes ou variáveis este sempre é o caso, se for uma função é possível pararmos no meio de um termo $t_m=ft_1'\dots t_p'$, nesse caso se $\xi$ terminar antes de $t_q'$ pegamos $t=t_{q+1}'$ e $\eta$ como o resto.
\end{proof}

\hrule

\textbf{Exercício 5.2.}
Mostre que o cálculo $\mf{C}_{nf}$ permite derivar precisamente aquelas strings da forma $x$ $\varphi$ no qual $\varphi\in\mc{L}^\mc{S}$ tq $x\notin\msf{free}(\varphi)$:\\\\
\infer[\text{Se }t_1,t_2\in\mc{T}^\mc{S}\text{ e }x\notin\msf{var}(t_1)\cup\msf{var}(t_2);]{x~~~t_1\overtext{.}{=} t_2}{\phantom{aaaaaaa}}\\\\
\infer[\text{Se }R\in\mc{S}\text{ é n-ária},t_1,\dots,t_n\in\mc{T}^\mc{S}\text{ e }x\notin\bigcup_{n\in\mbb{N}}\msf{var}(t_n);]{x~~~Rt_1\dots t_n}{\phantom{aaaaaaa}}\\\\
\infer[;]{x~~~\neg\varphi}{x & \varphi}~~~~
\infer[*=\wedge,\vee,\to,\sse;]{x~~~(\varphi*\psi)}{(x & \varphi) & (x & \psi)}~~~~
\infer[;]{x~~~Qx\varphi}{\phantom{aaaaaaa}}~~~~
\infer[Q=\forall,\exists;]{x~~~Qx\varphi}{x & \varphi}

\begin{proof}
    $(\Rightarrow)$ Fazendo indução em cada regra:\\
    $\varphi=t_1\overtext{.}{=} t_2$: por definição $x\notin\msf{free}(\varphi)$;\\
    $\varphi=Rt_1\dots t_n$: Também por definição $x\notin\msf{free}(\varphi)$;\\
    $\varphi=Qx\psi$ nesse caso $x\notin\msf{free}(\varphi)=\msf{free}(\psi)\backslash\{x\}$;\\
    (*) Portanto todas as fórmulas $\varphi$ deriváveis com premissa livre não tem uma ocorrência livre de $x$.\\
    $\varphi=\neg\psi$: Se $\neg\psi$ é derivável, então $\psi$ também é, mas se $\psi$ é derivável em $\mf{C}_{nf}$ então, por (*), $x\notin\msf{free}(\psi)\to x\notin\msf{free}(\neg\psi)$;\\
    $\varphi=(\psi*\chi)$: O argumento é análogo ao de cima, ambos $\psi,\chi$ tem de ser derivável e, por (*), não há ocorrência livre neles, o que implica que não há em $(\psi*\chi)$.\\\\
    $(\Leftarrow)$ Agora assumindo $x\notin\msf{free}(\varphi)$:\\
    $\varphi=t_1\overtext{.}{=} t_2$: então ela é derivável pela regra 1;\\
    $\varphi=Rt_1\dots t_n$: então ela é derivável pela 2ª regra;\\
    $\varphi=Qx\psi$: a última e penúltima regra garantem que é derivável;\\
    $\varphi=\neg\psi$: então $x\notin\msf{free}(\varphi)$, portanto a 3ª regra garante que é derivável;\\
    $\varphi=(\psi*\chi)$: Se $x$ não ocorre livre em $\varphi$ então ela não ocorre livre em ambos, portanto a 5ª regra garante sua derivação.
\end{proof}

\section{Semântica das Linguagens de Primeira Ordem}

\textbf{Exercício 1.4.} Seja $\mf{I}:=(\mf{A},\beta)$ tq $\mf{A}:=(\mbb{N},+,\cdot,0,1,<)$ e $\beta(v_n):=2n$ para $n\ge0$. Interprete as seguintes fórmulas:\\
a) $\exists v_0v_0+v_0\overtext{.}{=} v_1$;\\
b) $\exists v_0v_0\cdot v_0\overtext{.}{=} v_1$;\\
c) $\exists v_1v_0\overtext{.}{=} v_1$;\\
d) $\forall v_0\exists v_1v_0\overtext{.}{=} v_1$;\\
e) $\forall v_0\forall v_1\exists v_2(v_0<v_2\wedge v_2<v_1)$.

\begin{proof}
    $\mf{I}\vDash a)$ sse há um $a\in\mbb{N}$ tq $a+a=\beta(v_1)=2$, de fato $a=1$ satisfaz;\\
    $\mf{I}\vDash b)$ sse há um $a\in\mbb{N}$ tq $a\cdot a=2$, obviamente a equação $x^2=2$ não tem solução nos naturais, portanto $\mf{I}\nvDash b)$;\\
    $\mf{I}\vDash c)$ sse há um $a\in\mbb{N}$ tq $0=a$, o que é claramente verdade;\\
    $\mf{I}\vDash d)$ sse para todo $a\in\mbb{N}$ existe um $b\in\mbb{N}$ tq $a = b$, o que também é verdadeiro;\\
    $\mf{I}\vDash e)$ sse para todo $a,b\in\mbb{N}$ existe um $c\in\mbb{N}$ tq $a<c$ e $c<b$. Em particular escolhendo $b = a+1$ temos que existe um natural $c$ tq $a < c < a+1$ o que é falso, portanto $\mf{I}\nvDash e)$.
\end{proof}

\hrule

\textbf{Exercício 1.5.} Seja $A\ne\varnothing$ e $A,\mc{S}\prec\aleph_0$ um conjunto de símbolos. Mostre que há uma quantidade finita de $\mc{S}$-estruturas com domínio $A$.

\begin{proof}
Seja $S=((c_i)_{0\le i\le n_1},(R_i)_{0\le i\le n_2},(f_i)_{0\le i\le n_3})$ e $|A|=m$, a quantidade total de associações possíveis para cada elemento é:
\begin{align*}
    \alpha_{R_i} :=\{Z\mid Z\subseteq A^n\}, & ~~~~|\alpha_{R_i}| =\mc{P}(\alpha_{R_i}) = 2^m\\
    \alpha_{f_i} :=A^{\left(A^n\right)}, & ~~~~|\alpha_{f_i}| = |A|^{|A^n|} = m^{\left(m^n\right)}\\
    \alpha_{c_i} :=\left(A^n\right)^{A^n}, & ~~~~|\alpha_{c_i}| = |\left(A^n\right)|^{|A^n|} = \left(m^{n\cdot m^n}\right)
\end{align*}
Dessa forma, como todos são finitos e a união finita de conjuntos finitos é finita então o total de estruturas $\mc{H}$:
\[
\mc{H}:=\bigcup\Biggl\{\bigcup_{0\le i\le n_1}\alpha_{R_i},\bigcup_{0\le i\le n_2}\alpha_{f_i},\bigcup_{0\le i\le n_3}\alpha_{c_i}\Biggr\}\prec\aleph_0.
\]
\end{proof}

\hrule

\textbf{Exercício 1.6.}
Para $\mc{S}$-estruturas $\mf{A}=(A,\mf{a})$ e $\mf{B}=(B,\mf{b})$ seja $\mf{A}\times\mf{B}$ a $\mc{S}$-estrutura com domínio $A\times B$ satisfazendo:\\
Para $R\in\mc{S}$ n-ária e $(a_1,b_1),\dots,(a_n,b_n)\in A\times B$:
\[
R^{\mf{A}\times\mf{B}}(a_1,b_1)\dots(a_n,b_n)\sse R^\mf{A}a_1\dots a_n\wedge R^\mf{B}b_1\dots b_n;
\]
Para $f\in\mc{S}$ n-ária e $(a_1,b_1),\dots,(a_n,b_n)\in A\times B$:
\[
f^{\mf{A}\times\mf{B}}((a_1,b_1),\dots,(a_n,b_n)):=(f^\mf{A}(a_1,\dots,a_n),f^\mf{B}(b_1,\dots,b_n));
\]
Para $c\in\mc{S}$:
\[
c^{\mf{A}\times\mf{B}}:=(c^\mf{A},c^\mf{B});
\]
Mostre que:\\
(a) Se as $\mc{S}_{\msf{gr}}$-estruturas $\mf{A}$ e $\mf{B}$ são grupos então $\mf{A}\times\mf{B}$ também é.\\
(b) Se $\mf{A},\mf{B}$ são estruturas satisfazendo os axiomas de equivalência então $\mf{A}\times\mf{B}$ também satisfaz.\\
(c) Se as $\mc{S}_{\msf{ar}}$-estruturas $\mf{A},\mf{B}$ são corpos, então $\mf{A}\times\mf{B}$ não é.

\begin{proof}
(a) Sejam $\mf{A}=(A,\circ,e);\mf{B}=(B,*,\varepsilon)$ e $\mf{A}\times\mf{B}=(A\times B,\circledast,\epsilon)$. Se $a,b,c\in\mf{A};x,y,z\in\mf{B}$ e $u,v,w\in\mf{A}\times\mf{B}$:\\\\
(i) $\forall u,v,w((u\circledast v)\circledast w = u\circledast(v\circledast w))$:
\begin{align*}
    (\overbrace{(x,a)}^u\circledast \overbrace{(y,b)}^v)\circledast \overbrace{(z,c)}^w & = (x\circ y,a*b)\circledast (z,c)\\
    & = (x\circ y\circ z,a*b*c)\\
    & = (x,a)\circledast(y\circ z,b*c)\\
    & = (x,a)\circledast((y,b)\circledast(z,c))\\
    (u\circledast v)\circledast w & = u\circledast(v\circledast w).
\end{align*}
(ii) $\forall u\exists v(u\circledast v)=\epsilon$:
\begin{align*}
    (\overbrace{(x,a)}^u\circledast\overbrace{(y,b)}^v) & = \overbrace{(e,\varepsilon)}^\epsilon\\
    (x\circ y,a*b) & =(e,\varepsilon)\\
    \forall x\exists y(x\circ y=e)& \wedge \forall a\exists b(a*b=\varepsilon).
\end{align*}
(iii)$\exists\epsilon\forall u(u\circledast\epsilon=u)$:
\begin{align*}
    (\overbrace{(x,a)}^u\circledast\overbrace{(e,\varepsilon)}^\epsilon) & =\overbrace{(x,a)}^u\\
    (x\circ e,a\circ\varepsilon) & = (x,a)\\
    \exists e\forall x(x\circ e=x) & \wedge \exists\varepsilon\forall a(a*\varepsilon=a).
\end{align*}
(b) Sejam $\mf{A}=(A,R);\mf{B}=(B,\mc{R}),\mf{A}\times\mf{B}=(A\times B,\mathscr{R})$ com $x,y,z\in\mf{A};a,b,c\in\mf{B};u,v,w\in\mf{A}\times\mf{B}$:\\\\
(i) $\forall u(u\mathscr{R}u)$:
\begin{align*}
    \overbrace{(x,a)}^u\mathscr{R}\overbrace{(x,a)}^u & \sse xRx\wedge a\mc{R}a\\
    \forall x(xRx) & \wedge \forall a(a\mc{R}a).
\end{align*}
(ii) $\forall u,v(u\mathscr{R}v\sse v\mathscr{R}u)$:
\begin{align*}
    \overbrace{(x,a)}^u\mathscr{R}\overbrace{(y,b)}^v & \sse \overbrace{(y,b)}^v\mathscr{R}\overbrace{(x,a)}^u\\
    xRy\wedge a\mc{R}b & \sse yRx\wedge b\mc{R}a\\
    \forall x,y(xRy\sse yRx) & \wedge \forall a,b(a\mc{R}b\sse b\mc{R}a)
\end{align*}
(iii) $\forall u,v,w(u\mathscr{R} v\wedge v\mathscr{R} w\to u\mathscr{R}w)$:
\begin{align*}
    \overbrace{(x,a)}^u\mathscr{R}\overbrace{(y,b)}^v\wedge \overbrace{(y,b)}^v\mathscr{R}\overbrace{(z,c)}^w & \to \overbrace{(x,a)}^u\mathscr{R}\overbrace{(z,c)}^w\\
    (xRy\wedge a\mc{R}b)\wedge(yRz\wedge b\mc{R}c) & \to xRz\wedge a\mc{R}c\\
    (xRy\wedge yRz)\wedge(a\mc{R}b\wedge b\mc{R}c) & \to xRz\wedge a\mc{R}c\\
    \forall x,y,z(xRy\wedge yRz\to xRz) & \wedge\forall a,b,c(a\mc{R}b\wedge b\mc{R}c\to a\mc{R}c)
\end{align*}
(c) Sejam $\mf{A}=(A,+,\cdot,0,1);\mf{B}=(B,*,\times,\ol{0},\ol{1})$ e $\mf{A}\times\mf{B}=(A\times B,\oplus,\odot,\mathbf{0},\mathbf{1})$ com $x,y\in\mf{A};a,b\in\mf{B}$ e $u,v\in\mf{A}\times\mf{B}$:\\\\
Um dos axiomas é $\forall(u\ne\mathbf{0})\exists v(u\oplus v=\mathbf{1})$:
\begin{align*}
    (x,a)\oplus(y,b) & =(1,\ol{1})\\
    (x\cdot y,a*b) & = (1,\ol{1})
\end{align*}
Se isso é verdade então, em particular, para ou $x=0$ ou $a=\ol{0}$ temos que $(0,b),(x,\ol{0})\ne\mathbf{0}$, logo ambos $0,\ol{0}$ possuiriam invreso, o que é falso.
\end{proof}

\hrule

\textbf{Exercício 2.1.} Mostre que para $x,y\in\{\top,\bot\}$:\\
a) $\overtext{.}{\to}(x,y)=\overtext{.}{\vee}(\overtext{.}{\neg}(x),y)$;\\
b) $\overtext{.}{\wedge}(x,y)=\overtext{.}{\neg}(\overtext{.}{\vee}(\overtext{.}{\neg}(x),\overtext{.}{\neg}(y)))$;\\
c) $\overtext{.}{\leftrightarrow}(x,y)=\overtext{.}{\wedge}(\overtext{.}{\to}(x,y),\overtext{.}{\to}(y,x))$.

\begin{proof}
    \begin{tabular}{||c c c c c||} 
     \hline
     $x$ & $y$ & $\overtext{.}{\neg}(x)$ & $\overtext{.}{\vee}(\overtext{.}{\neg}(x),y)$ & $\overtext{.}{\to}(x,y)$ \\
     \hline\hline
     $\top$ & $\top$ & $\bot$ & $\top$ & $\top$\\ 
     \hline
     $\top$ & $\bot$ & $\bot$ & $\bot$ & $\bot$\\
     \hline
     $\bot$ & $\top$ & $\top$ & $\top$ & $\top$\\
     \hline
     $\bot$ & $\bot$ & $\top$ & $\top$ & $\top$\\
     \hline
    \end{tabular}

    \begin{tabular}{||c c c c c c c||} 
     \hline
     $x$ & $y$ & $\overtext{.}{\neg}(x)$ & $\overtext{.}{\neg}(y)$ & $\overtext{.}{\vee}(\overtext{.}{\neg}(x),\overtext{.}{\neg}(y))$ & $\overtext{.}{\neg}(\overtext{.}{\vee}(\overtext{.}{\neg}(x),\overtext{.}{\neg}(y)))$ & $\overtext{.}{\wedge}(x,y)$\\
     \hline\hline
     $\top$ & $\top$ & $\bot$ & $\bot$ & $\bot$ & $\top$ & $\top$\\ 
     \hline
     $\top$ & $\bot$ & $\bot$ & $\top$ & $\top$ & $\bot$ & $\bot$\\
     \hline
     $\bot$ & $\top$ & $\top$ & $\bot$ & $\top$ & $\bot$ & $\bot$\\
     \hline
     $\bot$ & $\bot$ & $\top$ & $\top$ & $\top$ & $\bot$ & $\bot$\\
     \hline
    \end{tabular}

    \begin{tabular}{||c c c c c c||} 
     \hline
     $x$ & $y$ & $\overtext{.}{\to}(x,y)$ & $\overtext{.}{\to}(y,x)$ & $\overtext{.}{\wedge}(\overtext{.}{\to}(x,y),\overtext{.}{\to}(y,x))$ & $\overtext{.}{\leftrightarrow}(x,y)$\\
     \hline\hline
     $\top$ & $\top$ & $\top$ & $\top$ & $\top$ & $\top$\\ 
     \hline
     $\top$ & $\bot$ & $\bot$ & $\top$ & $\bot$ & $\bot$\\
     \hline
     $\bot$ & $\top$ & $\top$ & $\bot$ & $\bot$ & $\bot$\\
     \hline
     $\bot$ & $\bot$ & $\top$ & $\top$ & $\top$ & $\top$\\
     \hline
    \end{tabular}
\end{proof}

\hrule

\textbf{Exercício 3.3.} Seja $P$ um símbolo de relação unária e $f$ de função binária. Determine duas interpretações para cada fórmula uma que a satisfaça e outra que não:\\
a) $\forall v_1fv_0v_1\overtext{.}{=}v_0$;\\
b) $\exists v_0\forall v_1fv_0v_1\overtext{.}{=}v_1$;\\
c) $\exists v_0(Pv_0\wedge\forall v_1Pfv_0v_1)$.

\begin{proof}
    a) Seja $\mf{I}=(\mbb{N},R,\cdot)$ tq $\beta(v_0)=0$, então $\mf{I}\vDash a)$ sse para todo $n\in\mbb{N}$ vale $0\cdot a=0$, o que é fato. Entretanto para mesma interpretação com $+$ temos $a+0=0$, o que não é o caso.\\
    b) Interpretando com a mesma estrutura que em a) o que b) garante é a existência de um elemento neutro, o que é verdade. Pro caso de não satisfação basta retirarmos o elemento neutro do domínio.\\
    c) Seja $Rx:=$ x é par para mesma estrutura $\mf{I}$ com $+$, o que c) diz é que existe um $x$ par tq para todo $y$, $x+y$ é par, o que é claramente falso, use, entretanto, $\cdot$ ao invés de $+$, então obviamente para todo $y$, $xy$ é par se $x$ for par.
\end{proof}

\hrule

\textbf{Exercício 3.4.} Uma fórmula sem $\neg,\to$ e $\leftrightarrow$ é denominada \textit{positiva}. Prove que toda fórmula positiva é satisfatível.

\begin{proof}
    Seja $\mf{I}=(\mf{A},\beta)$ tq $\msf{Dom}(\mf{A})=\{a\}$ e $\beta(v)=a$, com $R_i^\mf{A}$ sendo o grafo da função identidade n-ária para todo $i$, assim como $f_i^\mf{A}=\text{id}$ e $c_i^\mf{A}=a$. De fato, $\mf{I}(t)=a,\forall t \in\mc{T}^\mc{S}$. Por indução em fórmulas é claro que $\mf{I}\vDash t_1\equiv t_2$ e $\mf{R}t_1\dots t_n$, logo também satisfaz $\varphi\wedge\psi$ e $\varphi\vee\psi$, o mesmo para $\forall x\varphi$ e $\exists x \varphi$.
\end{proof}

\hrule

\textbf{Exercício 4.9.}
Para fórmulas arbitrárias $\varphi,\psi,\chi$ prove que:\\
a) $(\varphi\vee\psi)\vDash\chi$ sse $\varphi\vDash\chi$ e $\psi\vDash\chi$;\\
b) $\vDash(\varphi\to\psi)$ sse $\varphi\vDash\psi$.

\begin{proof}
    a) ($\Rightarrow$) Basta provarmos que $(\varphi\vee\psi\rightarrow\chi)\vDash\Dashv((\varphi\rightarrow\chi)\wedge(\psi\rightarrow\chi))$, logo
    \begin{align*}
        \varphi\vDash\chi\text{ e }\psi\vDash\chi &\text{ sse para todo }\mf{I}\text{, se }\mf{I}\vDash\varphi\text{, então }\mf{I}\vDash\chi\text{, i.e., }\mf{I}\vDash(\varphi\rightarrow\chi)\text{ e, igualmente, }\mf{I}\vDash(\psi\rightarrow\chi);\\
        &\text{ sse }\mf{I}\vDash((\varphi\rightarrow\chi)\wedge(\psi\rightarrow\chi));\\
        &\text{ sse }\mf{I}\vDash(\varphi\vee\psi\rightarrow\chi);\\
        &\text{ sse }(\varphi\vee\psi)\vDash\chi.
    \end{align*}
    b) ($\Leftarrow$)
    \begin{align*}
        \varphi\vDash\psi & \text{ sse para todo }\mf{I}\text{ se }\mf{I}\vDash\varphi\text{ então }\mf{I}\vDash\psi;\\
        & \text{ sse para todo }\mf{I}\vDash(\varphi\to\psi);\\
        & \text{ sse }\vDash(\varphi\to\psi).
    \end{align*}
\end{proof}

\hrule

\textbf{Exercício 4.10.}
Mostre que:\\
(a) $\exists x\forall y\varphi\vDash\forall y\exists x\varphi;$\\
(b) $\forall y\exists xRxy~\nvDash~\exists x\forall yRxy$.

\begin{proof}
(a) $\mf{I}\vDash\exists x\forall y\varphi$ sse existe um $a\in A$ tq $\mf{I}\frac{a}{x}\vDash\forall y\varphi$, então em particular existe um $a\in A$ tq $\mf{I}\frac{a}{x}\frac{t}{y}\vDash\varphi$ sendo $t\in A$ um termo genérico qualquer. Assim, devido a escolha arbitrária, concluímos que para todo $t\in A$ existe um $a\in A$ tq $\mf{I}\frac{a}{x}\frac{t}{y}\vDash\varphi$, i.e., $\mf{I}\vDash\forall y\exists x\varphi$.\\
(b) $\mf{I}\vDash\forall y\exists xRxy$ sse para todo $a\in A$ existe um $t\in A$ tq $\mf{I}\vDash Rta$, mas isso não necessariamente implica que exista um $t$ tq $Rta$ valha para todo $a$.\\
\textbf{Obs:} Lembre-se que a definição de satisfatibilidade é feita na metateoria que, por mais rigorosa que seja, é justificada pela noção intuitiva que temos de cada fórmula e justificada da mesma forma.
\end{proof}

\hrule

\textbf{Exercício 4.11.} Prove que para $Q=\forall,\exists$:\\
a) $Qx(\varphi\wedge\psi)\vDash\Dashv(Qx\varphi\wedge Qx\psi)$;\\
b) $Qx(\varphi\vee\psi)\vDash\Dashv(\varphi\vee Qx\psi)$, se $x\notin\msf{free}(\varphi)$;\\
e justifique o motivo da assunção $x\notin\msf{free}(\varphi).$

\begin{proof}
    Provarei para $Q=\forall$ porque é fácil ver que a intuição se estende pro outro caso.\\
    a) Obviamente se para todo $a\in A$ temos $\mf{I}\frac{a}{x}\vDash\varphi$ e para todo $b\in A$ temos $\mf{I}\frac{b}{x}\vDash\psi$, então para todo $c\in A$, $\mf{I}\frac{c}{x}\vDash\varphi$ e $\mf{I}\frac{c}{x}\vDash\psi$, i.e., para todo $c\in A$, $\mf{I}\vDash(\varphi\wedge\psi)$, analogamente vale a volta. A justificativa se baseia no fato intuitivo de que se estamos variando pelo domínio todo de uma forma numa fórmula e de outra forma na outra, então podemos variar em ambas da mesma forma.\\
    b) $\mf{I}\vDash\forall x\varphi$ sse para todo $a\in A$, $\mf{I}\frac{a}{x}\vDash\varphi$, i.e., utilizamos a valoração $\beta$ que interpreta $x$ como $a$, mas como $x\notin\msf{free}(\varphi)$, então $\mf{I}\frac{a}{x}(\varphi)=\mf{I}(\varphi)$, a partir disso é fácil provar ambos b) e c).
\end{proof}

\hrule

\textbf{Exercício 4.12.}
Sejam $\varphi,\psi$ fórmulas tais que $\varphi\vDash\Dashv\psi$. Seja $\chi'$ obtido de $\chi$ substituindo todas as subfórmulas da forma $\varphi$ por $\psi$. Mostre que para todo $\chi,\chi\vDash\Dashv\chi'$.

\begin{proof}
Provaremos por indução em fórmulas:\\
Se $\chi=\varphi$ é atômica então $\mf{I}\vDash\varphi$ sse, por hipótese, $\mf{I}\vDash\chi'=\psi$;\\
se $\chi=\neg\varphi$ então $\mf{I}\vDash\chi$ sse não vale $\mf{I}\vDash\varphi$ sse, por hipótese, não vale $\mf{I}\vDash\psi$, i.e., $\mf{I}\vDash\chi'=\neg\psi$;\\
se $\chi=\xi\vee\varphi$ então $\mf{I}\vDash\chi$ sse $\mf{I}\vDash\xi$ ou $\mf{I}\vDash\varphi$ sse, por hipótese, $\mf{I}\vDash\xi$ ou $\mf{I}\vDash\psi$, i.e., $\mf{I}\vDash\chi'=\xi\vee\psi$;\\
se $\chi=\exists x\varphi$ então $\mf{I}\vDash\chi$ sse existe um $a\in A$ tq $\mf{I}\frac{a}{x}\vDash\varphi$ sse, por hipótese, existe um $a\in A$ tq $\mf{I}\frac{a}{x}\vDash\psi$, i.e., $\mf{I}\vDash\chi'=\exists x\psi$.\\
Portanto $\chi\vDash\Dashv\chi'$.
\end{proof}

\hrule

\textbf{Exercício 4.13.}
Prove o análogo ao \textbf{4.8.} para relação de consequência.

\begin{proof}
    Pelo \textbf{Lema 4.4.} é fácil estender o caso que o conjunto é satisfatível para consequência lógica.
\end{proof}

\hrule

\textbf{Exercício 4.14.}
Um conjunto $\Phi$ de sentenças é dito \textit{independente} se não há um $\varphi\in\Phi$ tq $\Phi\backslash\{\varphi\}\vDash\varphi$. Mostre que os conjuntos $\Phi_\text{gr}$ e $\Phi_\text{eq}$ de axiomas dos grupos e relações de equivalência são independentes.

\begin{proof}
(a) $\Phi_\text{gr} = \{\underbrace{\forall uvw((u\circ v)\circ w=u\circ(v\circ w))}_{\varphi_1},\underbrace{\forall u\exists v(u\circ v=e)}_{\varphi_2},\underbrace{\exists c\forall u(u\circ c=u)}_{\varphi_3}\}$\\
(i) Como $\varphi_3$ garante a existência de um elemento neutro, mas não necessariamente precisamos interpretar $e$ como este, peguemos $(\mbb{N}\backslash\{1\},\cdot,0)$, de fato esta é associativa e possui um número que se operado com qualquer outro no domínio resulta em $0$, sendo este, é claro, também o $0$, então $\mf{I}\vDash\Phi_\text{gr}\backslash\{\varphi_3\}$, mas $\mf{I}\nvDash\varphi_3$; \\
(ii) Como $\varphi_2$ garante a existência de um inverso, basta tomarmos a estrutura $(\mbb{N},+,0)$ em $\mf{I}$ que vale $\mf{I}\vDash\Phi_\text{gr}\backslash\{\varphi_2\}$, mas $\mf{I}\nvDash\varphi_2$;\\
(iii) Como $\varphi_1$ garante associatividade tomamos o operador $\circ$ como não associativo, por exemplo a estrutura $(\mbb{Z},-,0)$ em $\mf{I}$ garante que $\mf{I}\vDash\Phi_\text{gr}\backslash\{\varphi_1\}$, mas $\mf{I}\nvDash\varphi_1$.\\\\
(b) $\Phi_\text{eq} = \{\underbrace{\forall a(aRa)}_{\varphi_1},\underbrace{\forall ab(aRb\sse bRa)}_{\varphi_2},\underbrace{\forall abc(aRb\wedge bRc\to aRc)}_{\varphi_3}\}$\\
(i) Para $\Phi_\text{eq}\backslash\{\varphi_3\}$ basta tomar $(\mbb{Z},\cdot,R)$ tq $aRb$ sse $a\cdot b \ge 0$. Assim ambos $\varphi_1,\varphi_2$ são satisfeitos, mas escolhendo $b=0$ em $\varphi_3$ tal relação não é sempre verdade;\\
(ii) Para $\Phi_\text{eq}\backslash\{\varphi_2\}$ basta tomar $(\mbb{N},\ge)$, tal qual não é simétrica;\\
(iii) Para $\Phi_\text{eq}\backslash\{\varphi_1\}$ basta tomar $A=\{a\}$ e $(A,R)$ tq $\forall a\in A(a\cancel{R}a)$.
\end{proof}

\hrule

\textbf{Exercício 4.15.} (Generalização do \textbf{Exercício 1.6.}). Seja $I\ne\varnothing$, $\forall i\in I$, seja $\mf{A}_i$ uma $\mc{S}$-estrutura. Denotaremos por $\prod_{i\in I}\mf{A}_i$ a $\mc{S}$-estrutura do produto direto das $\mc{S}$-estruturas $\mf{A}_i$:
$$\msf{Dom}\left(\prod_{i\in I}\mf{A}_i\right):=\left.\biggl\{g~\right|~ g:I\to\bigcup_{i\in I}\msf{Dom}(\mf{A}_i)\text{, e }g(i)\in\msf{Dom}(\mf{A}_i),\forall i\in I\biggr\}$$
i.e., n-tuplas de todas as possíveis combinações de elementos no domínio de cada estrutura (que denotaremos por $\langle g(i) \mid i \in I\rangle$), e:\\
para $R\in\mc{S}$ n-ária e $g_1,\dots,g_n\in\prod_{i\in I}\msf{Dom}(\mf{A}_i):$
$$R^\mf{A}g_1\dots g_n\text{ sse }R^{\mf{A}_i}g_1(i)\dots g_n(i), \forall i\in I;$$
para $f\in\mc{S}$ n-ária e $g_1,\dots,g_n\in\prod_{i\in I}\msf{Dom}(\mf{A}_i):$
$$f^\mf{A}(g_1,\dots,g_n):=\langle f^{\mf{A}_i}(g_1(i),\dots,g_n(i)) \mid i\in I\rangle;$$
e $c^\mf{A}:=\langle c^{\mf{A}_i}\mid i\in I\rangle$ para $c\in\mc{S}$.\\
Prove que para $t\in\mc{T}^\mc{S}$ se $\msf{var}(t)\subseteq\{v_0,\dots,v_{n-1}\}$ e $g_0,\dots,g_{n-1}\in\prod_{i\in I}\msf{Dom}(\mf{A}_i)$, então
$$t^\mf{A}[g_0,\dots,g_{n-1}]=\langle t^{\mf{A}_i}[g_0(i),\dots,g_{n-1}(i)]\mid i\in I\rangle~~(*)$$

\begin{proof}
    Se $t=c$, então, por definição, $c^\mf{A}=\langle c^{\mf{A}_i}\mid i\in I\rangle$. Se $t=x$, então, novamente por definição, $t^\mf{A}[g_0]=g_0=\langle g_0(i)\mid i\in I\rangle$. Provados os casos bases assuma $(*)$ como hipótese indutiva. Se $t=f(t_1,\dots,t_n)$, então $t^\mf{A}=f^\mf{A}(t_1^\mf{A},\dots,t_n^\mf{A})$, por hipótese para cada $t_i$ temos $t_i^\mf{A}=g_k$, para algum $g_k$, logo $t^\mf{A}=f^\mf{A}(g_{i_1},\dots,g_{i_n})$ que, por definição, é igual a $\langle f^{\mf{A}_i}(g_{i_1}(i),\dots,g_{i_n}(i))\mid i\in I\rangle$.
\end{proof}

\hrule

\textbf{Exercício 4.16.} Fórmulas deriváveis no seguinte cálculo são denominadas fórmulas Horn:\\\\
\infer[\text{Se }n\in\mbb{N}\text{ e }\varphi_1,\dots,\varphi_n,\varphi\text{ são atômicas};]{(\neg\varphi_1\vee\dots\vee\neg\varphi_n\vee\varphi)}{}\\\\
\infer[\text{Se }n\in\mbb{N}\text{ e }\varphi_0,\dots,\varphi_n\text{ são atômicas};]{\neg\varphi_0\vee\dots\vee\neg\varphi_n}{}\\\\
\infer[;]{(\varphi\wedge\psi)}{\varphi,\psi}~~~~
\infer[;]{\forall x\varphi}{\varphi}~~~~
\infer[.]{\exists x\varphi}{\varphi}\\
Mostre que se $\varphi$ é uma \textit{sentença} Horn e se $\mf{A}_i\vDash\varphi,\forall i\in I$, então $\prod_{i\in I}\mf{A}_i\vDash\varphi$.

\begin{proof}
    Pelo teorema anterior temos $\prod_{i\in I}\mf{A}_i\vDash(t_1\overtext{.}{=}t_2)$ sse $t_1^\mf{A}=\langle t_1^{\mf{A}_i}\mid i\in I\rangle=t_2^\mf{A}=\langle t_2^{\mf{A}_i}\mid i\in I\rangle$, i.e., $t_1^{\mf{A}_i}=t_2^{\mf{A}_i},\forall i\in I$, então obviamente se cada $\mf{A}_i$ o satisfaz, o produto direto também. É fácil estender o argumento paras outras fórmulas atômicas. Disso é fácil tirar que se todas as estruturas satisfazem negações e disjunções de fórmulas atômicas, então o produto direto também satisfaz.\\
    Provado o caso base assuma como hipótese de indução que se $\mf{A}_i\vDash\varphi,\forall i\in I$, então $\prod_{i\in I}\mf{A}_i\vDash\varphi$. Se $\mf{A}_i\vDash(\varphi\wedge\psi)$, então $\mf{A}_i\vDash\varphi$ e $\mf{A}_i\vDash\psi$, por hipótese isso implica que $\prod_{i\in I}\mf{A}_i\vDash\varphi$ e $\prod_{i\in I}\mf{A}_i\vDash\psi$, i.e., $\prod_{i\in I}\mf{A}_i\vDash(\varphi\wedge\psi)$. Da mesma forma, $\mf{A}_i\vDash\exists x\varphi$ sse existe $a\in\msf{Dom}(\mf{A}_i)$ tq $\mf{A}_i\frac{a}{x}\vDash\varphi$, se em cada domínio das $\mf{A}_i$ há um elemento que satisfaz, em particular pegando $a_i\in\msf{Dom}(\mf{A}_i)$ temos que a n-tupla $(a_1,\dots,a_n)\in\msf{Dom}\left(\prod_{i\in I}\mf{A}_i\right)$ também satisfaz, o argumento é análogo para $\forall x\varphi$.
\end{proof}

\hrule

\textbf{Exercício 5.9.} Seja $\mc{S}\prec\aleph_0$ um conjunto de símbolos e $\mf{A}$ uma $\mc{S}$-estrutura tq $\msf{Dom}(\mf{A})\prec\aleph_0$. Mostre que há $\varphi_\mf{A}\in\mc{L}_0^\mc{S}$ cujos modelos são exatamente aquelas $\mc{S}$-estruturas isomórficas a $\mf{A}$.

\begin{proof}
    Construiremos $\varphi_\mf{A}$ em função de $\mf{A}$, enumere $\msf{Dom}(\mf{A})=\{a_0,\dots,a_{n-1}\}$. Como $\mc{S}\prec\aleph_0$, então para especificamente $x_1,\dots,x_n\in\msf{Var}$ defina $\Phi:=\{\varphi\mid \varphi\text{ é atômica e }\msf{free}(\varphi)=\{x_1,\dots,x_n\}\}$ o conjunto de $\mc{S}$-fórmulas atômicas com exatamente $x_1,\dots,x_n$ como variáveis livres. Obviamente $\Phi\prec\aleph_0$, enumere portanto $\Phi=\{\varphi_1,\dots,\varphi_k\}$. Defina por indução $\Psi_0:=\varnothing$ e\\
    $$\Psi_m:=
    \begin{cases}
        \Psi_{m-1}\cup\{\varphi_m\}\text{, se }\mf{A}\vDash\varphi_m[a_1,\dots,a_n];\\
        \Psi_{m-1}\cup\{\neg\varphi_m\}\text{, se } \mf{A}\nvDash\varphi_m[a_1,\dots,a_n].
    \end{cases}$$
    com isso o conjunto $$\Psi:=\bigcup_{i=1}^{k}\Psi_i$$ tem cardinalidade igual a $\Phi$ e, portanto, é finito. Obviamente $\Psi$ possui todas as informações necessárias para definirmos todas as funções, relações e constantes e suas dependências com os elementos do domínio, portanto toda estrutura que satisfaz $\Psi$ terá tais propriedades, basta agora garantir que o domínio dessa nova estrutura esteja em bijeção com o de $\mf{A}$, defina então:
    $$\varphi_\mf{A}:=\exists x_1\dots x_n\left(\bigwedge\Psi\wedge\forall x\left(\bigvee_{i=1}^n x\overtext{.}{=}x_i\right)\right)$$
\end{proof}

\hrule

\textbf{Exercício 5.10.}
Mostre que: (a) A relação $<$ é elementarmente definível em $(\mbb{R},+,\cdot,0)$, i.e., existe uma fórmula $\varphi\in\mc{L}^{\{+,\cdot,0\}}_2$ tq $\forall a,b\in\mbb{R}$:
\[
(\mbb{R},+,\cdot,0)\vDash\varphi[a,b]\text{ sse }a<b.
\]
(b) A relação $<$ não é elementarmente definível em $(\mbb{R},+,0)$.

\begin{proof}
(a) Tome $\varphi=\exists c(\neg(c\overtext{.}{=}0)\wedge(b\overtext{.}{=}a+c^2))$, dessa forma $(\mbb{R},+,\cdot,0)\vDash\varphi[a,b]$ sse $a<b$.\\
(b) Seja $\pi:\mf{A}\cong\mf{A}$ um automorfismo em $\mf{A}=(\mbb{R},+,\cdot,0)$ tq $\pi(a) = -a$ que é o $c\in\mbb{R}$ tq $a+c=0$. Para provar que $\pi$ é um automorfismo precisamos:\\
(i) $\pi$ é uma bijeção;\\
(ii) $\pi(a+b)=\pi(a)+\pi(b)$;\\
(iii) $\pi(0)=0$.\\
Como todos são verficados isso garante que é um automorfismo. Agora vejamos que se existe um $\varphi[a,b]$ tq $\mf{A}\vDash\varphi[a,b]$ sse $a<b$ então como $\pi$ é estritamente decrescente, $\mf{A}\vDash\varphi[\pi(a),\pi(b)]$ sse $a>b$. Sabemos, também, pelo \textbf{Lema do Isomorfismo} que $\mf{A}\vDash\varphi[a,b]$ sse $\mf{A}\vDash\varphi[\pi(a),\pi(b)]$, i.e., $a<b$ sse $b<a$, o que é uma contradição, portanto não existe tal $\varphi[a,b]$ e, com isso, $<$ não é elementarmente definível.
\end{proof}

\hrule

\textbf{Exercício 5.11.} Alterando o cálculo das fórmulas universais substituindo o quantificador universal em (iii) por um existencial conseguimos o cálculo de fórmulas existenciais. Prove que: \\
a) A negação de uma sentença universal é logicamente equivalente a uma sentença existencial, e vice versa;\\
b) Se $\mf{A}\subseteq\mf{B}$ e $\varphi$ é uma sentença existencial, então $\mf{A}\vDash\varphi\implies\mf{B}\vDash\varphi$.

\begin{proof}
    a) Caso base para ambas: Se $\varphi$ é livre de quantificadores, obviamente $\neg\varphi$ também é, portanto se $\varphi$ é uma sentença universal, $\neg\varphi$ é existencial e vice versa. Tomemos como hipótese indutiva que se $\varphi$ é universal/existencial, então $\neg\varphi$ é existencial/universal. Se $\varphi=(\psi\wedge\chi)$, então $\neg\varphi$ é logicamente equivalente a $\neg\psi\vee\neg\chi$, assim como para $\varphi=(\chi\vee\psi)$ temos $\neg\psi\wedge\neg\chi$, por hipótese é fácil ver que a propriedade é preservada para ambos os casos. Da mesma forma se $\varphi=\forall x\psi$, então $\neg\varphi$ é logicamente equivalente a $\exists x\neg\varphi$, o caso contrário é análogo.\\
    b) Por a) sabemos que $\neg\varphi$ é logicamente equivalente a uma fórmula universal, se $\mf{A}\vDash\varphi$, então $\mf{A}\nvDash\neg\varphi$, pela contraposição do \textbf{Corolário 5.8.} temos que $\mf{B}\nvDash\neg\varphi$, i.e., $\mf{B}\vDash\varphi$.
\end{proof}

\hrule

\textbf{Exercício 6.7.} Formalize as seguintes declarações usando o conjunto de símbolos de 6.2.:\\
a) Todo real positivo tem uma raiz quadrada positiva;\\
b) Se $\rho$ é estritamente monótona, então $\rho$ é injetiva;\\
c) $\rho$ é uniformemente contínua em $\mbb{R}$;\\
d) para todo $x$, se $\rho$ é diferenciável em $x$, então $\rho$ é contínua em $x$.

\begin{proof}
    a) $\forall x\exists y(0<y\wedge y\cdot y\overtext{.}{=}x)$;\\
    b) $(\forall x\forall y(x<y\rightarrow\rho(x)<\rho(y))\vee\forall x\forall y(x<y\rightarrow\rho(y)<\rho(x)))\rightarrow\forall x\forall y(\rho(x)\overtext{.}{=}\rho(y)\rightarrow x\overtext{.}{=}y);$\\
    c) $\forall u(0<u\rightarrow\exists v(0<v\rightarrow\forall x\forall y(\Delta(x,y)<v\rightarrow\Delta(\rho(x),\rho(y))<u)));$\\
    d) Sejam $$C(x):=\forall u(0<u\rightarrow\exists v(0<v\rightarrow\forall y(\Delta(y,x)<v\rightarrow\Delta(\rho(y),\rho(x))<u)));$$
    $$L(\ell, f(y), p):=\forall u(0<u\rightarrow\exists v(0<v\rightarrow\forall y((0<\Delta(y,p)\wedge\Delta(y,p)<v)\rightarrow\Delta(f(y),\ell)<u).$$
    Logo $\forall z(\exists w(\rho(x+y)\overtext{.}{=}w\cdot y+\rho(x)\wedge\exists\ell(L(\ell,w,0)))\rightarrow C(x)).$
\end{proof}

\hrule

\textbf{Exercício 6.8.} Seja $S_\text{eq}=\{R\}$, formalize:\\
a) $R$ é uma relação de equivalência com no mínimo duas classes de equivalência;\\
b) $R$ é uma relação de equivalência com uma classe de equivalência contendo mais de um elemento.

\begin{proof}
    a) $\bigwedge\Phi_\text{eq}\wedge\exists a\exists b(Rab\wedge\exists c(\neg Rac))$;\\
    b) $\bigwedge\Phi_\text{eq}\wedge\exists a\exists b(Rab\wedge\neg(a\overtext{.}{=}b))$.
\end{proof}

\hrule

\textbf{Exercício 6.9.} Utilize o \textbf{Exercício 4.16.} para provar que:\\
a) Se para todo $i\in I$ a estrutura $\mf{A}_i$ é um grupo, então $\prod_{i\in I}\mf{A}_i$ é um grupo;\\
b) Nem a teoria da ordem, nem a dos corpos, pode ser axiomatizada por uma sentença de Horn.

\begin{proof}
    a) Seja $\mf{A}=\prod_{i\in I}\mf{A}_i$, vale que $\mf{A}\vDash\forall x(x\circ e\overtext{.}{=}x)$ sse para todo $g\in\msf{Dom}(\mf{A})$ temos $g\circ^\mf{A}e^\mf{A}=g$, i.e., $\langle g(i)\circ^{\mf{A}_i}e^{\mf{A}_i}\mid i\in I\rangle=\langle g(i)\mid i\in I\rangle$ que é igual sse $g(i)\circ^{\mf{A}_i}e^{\mf{A}_i}=g(i),\forall i\in I$ o que, por hipótese, é verdade. Destrinchando os axiomas de grupo desta forma é fácil mostrar que $\mf{A}\vDash\Phi_\text{gr}$.\\
    b) Assuma que $\varphi_\text{fd}=\bigwedge\Phi_\text{fd}$ a conjunção dos axiomas de corpos seja uma sentença Horn, pelo \textbf{Exercício 1.6.} o produto direto de duas estruturas de corpos $\mf{C}=\mf{A}\times\mf{B}$ não é um corpo e, pelo \textbf{Exercício 4.16.}, deveria ser. Contradição, então $\varphi_\text{fd}$ não é uma sentença Horn.\\
    Igualmente se $\varphi_\text{ord}=\bigwedge\Phi_\text{ord}$ é uma sentença de Horn, então se $\mf{A},\mf{B}$ são estruturas de ordem, $\mf{C}=\mf{A}\times\mf{B}$ precisa também ser. Note que $\mf{C}\vDash\forall xy(x<y\vee x\overtext{.}{=}y\vee y<x)$ sse para $x=(a,b)$ e $y=(p,q)$ temos $\forall(a,b)(p,q)((a<p\wedge b<q)\vee(a=p\wedge b=q)\vee(p<a\wedge q<b))$, entretando escolhendo $(a,b),(p,q)$ tq $a<p$ e $b>q$ temos $\mf{C}$ não o satisfaz, contradição.
\end{proof}

\hrule

\textbf{Exercício 6.10.} $M\subseteq\mbb{N}$ é denominado \textit{spectrum} se há um conjunto de símbolos $\mc{S}$ e uma $\mc{S}$-sentença $\varphi$ tq
$$M=\{n\in\mbb{N}\mid\varphi\text{ possui um modelo com exatamente }n\text{ elementos}\}.$$
Prove que é um spectrum: a) Todo $N\subseteq\mbb{N}\backslash\{0\}$ finito;\\
b) $\{n\in\mbb{N}\backslash\{0\}\mid(n\equiv 0~(\text{mod }m))\wedge m\ge1\}$;\\
c) $\{n^2\mid n\in\mbb{N}\backslash\{0\}\}$;\\
d) $\{n\in\mbb{N}\backslash\{0\}\mid n\text{ não é primo}\}$;\\
e) $\{n\in\mbb{N}\mid n\text{ é primo}\}$.

\begin{proof}
    Seja $\varphi_{\ge n}:=\bigwedge_{i,j\in\{1,\dots,n\}}\neg(v_i\overtext{.}{=}v_j)$, então $$\varphi_n:=\exists v_1\dots v_n\left(\varphi_{\ge n}\wedge\forall v\left(\bigvee_{i=1}^n v\overtext{.}{=} v_i\right)\right)$$ é a formalização de há exatamente $n$ elementos.\\
    a) Como $N\prec\aleph_0$ enumere $N=\{a_1,\dots,a_n\}$, logo podemos descrever $N=\bigl\{n\in\mbb{N}\mid\bigvee_{i=1}^n\varphi_{a_i}\bigr\}$.\\
    b) Pegue $\mc{S}=\{R\}$ e defina $$\varphi=\bigwedge\Phi_\text{eq}\wedge\exists v_1\dots v_m\left(\varphi_{\ge m}\wedge\forall v\left(\bigvee_{i=1}^mRvv_i\right)\right)$$
    Isso garante não só que $R$ é uma relação de equivalência como garante que o conjunto quociente $\mf{A}/R$ de qualquer modelo de $\varphi$ terá exatamente $m$ classes de equivalência, como todas possuem a mesma cardinalidade tem de ser possível particionar o domínio em $m$ conjuntos diferentes, i.e., ser um múltiplo de $m$.\\
    c) Seja $\mc{S}=\{R,f,g\}$ a ideia é formalizar $\psi$ tq $f,g:\msf{Dom}(\mf{A})\to R$, $\chi$ tq $(f(x),g(x))$ é injetivo e $\xi$ que é sobrejetivo, i.e.:
    \begin{align*}
        \psi & := \forall x(Rf(x)\wedge Rg(x));\\
        \chi & := \forall x\forall y((f(x)=f(y)\wedge g(x)=g(y))\rightarrow x=y);\\
        \xi & := \forall x\forall y((Rx\wedge Ry)\rightarrow\exists z(f(z)=x\wedge g(z)=y)).
    \end{align*}
    Logo, se $\mf{A}\vDash\varphi:=\psi\wedge\chi\wedge\xi$, então $\mf{A}$ possui uma bijeção de $\msf{Dom}(\mf{A})$ em $R^2$, i.e., a cardinalidade do domínio será o quadrado de um natural. Para provarmos que sempre haverá um modelo para cada quadrado perfeito contruiremos um modelo para $\varphi$. Seja $\msf{Dom}(\mf{A}):=\{1,\dots,m\}$ e defina $R:=\{1,\dots,p\}$, se $f(x)$ é o quociente de $x\in\msf{Dom}(\mf{A})$ por $p$ e $g(x)$ o resto, então $x=pf(x)+g(x)$ com $f,g$ unicamente determinados, então para cada $x$ no domínio existem $(f(x),g(x))\in R^2$ e vice versa.\\
    d) \\
    e) $\varphi:=\bigwedge\Phi_\text{ofd}\wedge\forall x(\neg(x\overtext{.}{=}x+1)\rightarrow x<x+1)$ garante, visto que todo corpo finito tem característica prima e, portanto, contém $p^n$ elementos, a última restrição garante que $n=1$. Seja $\mf{A}\vDash\varphi$ cujo domínio tem $p$ elementos. Assuma por contradição que existe $\mf{B}\vDash\varphi$ tq $n\ne1$, então $\exists a\notin\mbb{F}_p$, portanto $a\ne0$, a vista disso temos $a<a+1<\dots<a+p=a$, contradição, visto que $<$ é uma relação de ordem total.
\end{proof}

\hrule

\textbf{Exercício 7.5.} Prove que:\\
a) Se $\mf{A}=(A,+^A,\cdot^A,0^A,1^A)\vDash\Pi$ e se $\sigma^A:A\to A$, dada por $\sigma^A(a)=a+^A1^A$, então $(A,\sigma^A,0^A)\vDash$(P1)-(P3).\\
b) $\mf{N}=(\mbb{N},+,\cdot,0,1)$ é caracterizada por $\Pi$ até o isomorfismo.

\begin{proof}
    a) Interpretemos em $\mf{A}$ os 3 primeiros axiomas de $\Pi$:\\
    (i) $\forall x(\neg x+^A1^A\overtext{.}{=}0^A)$ sse $\forall x(\neg\sigma(x)\overtext{.}{=}0)$ (P1);\\
    (ii) $\forall xy(x+^A1^A\overtext{.}{=}y+^A1^A\rightarrow x\overtext{.}{=}y)$ sse $\forall xy(\sigma(x)\overtext{.}{=}\sigma(y)\rightarrow x\overtext{.}{=}y)$ (P2);\\
    (iii) $\forall X((X0^A\wedge\forall x(Xx\rightarrow Xx+^A1^A))\rightarrow\forall yXy)$ sse $\forall X((X0^A\wedge\forall x(Xx\rightarrow X\sigma(x)))\rightarrow\forall yXy)$ (P3).\\
    b) Seja $\mf{A}=(A,+^A,\cdot^A,0^A,1^A)\vDash\Pi$ para $\pi:\mf{N}\cong\mf{A}$ definimos indutivamente:\\
    $\pi(0)=0^A$;\\
    $\pi(x+1)=\pi(x)+^A1^A$.\\
    Demonstraremos agora que $\pi$ é bijetivo:\\
    Sobretividade: a definição garante o caso base, $0^A\in\msf{Im}(\pi)$. Assuma por hipótese $a=\pi(n)\in\msf{Im}(\pi)$, logo $a+^A1^A=\pi(n)+^A1^A=\pi(n)+^A\pi(1)$, por definição $a+^A1^A=\pi(n+1)\in\msf{Im}(\pi)$.\\
    Injetividade: Queremos provar que $\forall nm(n\ne m\rightarrow\pi(n)\ne\pi(m))$. Indução em $n$:\\
    Caso base: $n=0$ e $m\ne0$, em particular, assuma sem perda de generalidade, que $m=k+1$, logo $\pi(n)=0^A$ e $\pi(m)=\pi(k+1)$, pela primeira sentença em $\Pi$, $k+1\ne0$, portanto $\pi(m)=\pi(k+1)\ne\pi(0)=0^A=\pi(n)$.\\
    Provado o caso base assuma como hipótese de indução que $\forall m(n\ne m\rightarrow\pi(n)\ne\pi(m))$, façamos agora indução dupla, dessa vez em $m$:\\
    Caso base: $m=0$ e $n\ne0$, em especial $n=k+1$, a prova deste é análogo ao caso base em $n$.\\
    Hipótese indutiva: $n\ne m\rightarrow\pi(n)\ne\pi(m)$, sejam $n,m\ne0$, então $n=p+1$ e $m=q+1$, se $n\ne m$, i.e., $\neg(p+1=q+1)$, por 2 em $\Pi$, $p\ne q$ e, por hipótese, $\pi(p)\ne\pi(q)$, portanto, se $\pi(n)=\pi(p)+^A1^A=\pi(m)=\pi(q)+^A1^A$, também por 2 em $\Pi$ temos $\pi(p)=\pi(q)$, contradição, logo $\pi(n)\ne\pi(m)$.\\
    Se $\pi$ é isomorfismo, provemos que (i) $\pi(n+m)=\pi(n)+^A\pi(m)$ e (ii) $\pi(n\cdot m)=\pi(n)\cdot^A\pi(m)$:\\
    (i)\\
    Caso base: $\pi(m+0)=\pi(m)=\pi(m)+^A0^A=\pi(m)+^A\pi(0)$, pela propriedade 4 em $\Pi$.\\
    Assuma $\pi(n+m)=\pi(n)+^A\pi(m)$ como hipótese de indução:
    \begin{align*}
        \pi(m+(n+1)) & = \pi((m + n) + 1) & \text{(P5)};\\
        & = \pi(m + n)+^A1^A & \text{definção};\\
        & = (\pi(m)+^A\pi(n))+^A1^A & \text{passo indutivo};\\
        & = \pi(m)+^A(\pi(n)+^A1^A) & \text{(P5)};\\
        & = \pi(m)+^A\pi(n+1) & \text{definição}.
    \end{align*}
    (ii)\\
    Caso base: $\pi(m\cdot0)=\pi(0)=\pi(m)\cdot0^A=\pi(m)\cdot\pi(0)$, pela propriedade 6 em $\Pi$.\\
    Assuma $\pi(n\cdot m)=\pi(n)\cdot^A\pi(m)$ como hipótese e indução:
    \begin{align*}
        \pi(m\cdot(n+1)) & = \pi(m\cdot n + m) & \text{(P7)};\\
        & = \pi(m\cdot n)+^A\pi(m); & \\
        & = (\pi(m)\cdot^A\pi(n))+^A\pi(m) & \text{passo indutivo};\\
        & = \pi(m)\cdot^A(\pi(n)+^A1^A) & \text{(P7)};\\
        & = \pi(m)\cdot^A\pi(n+1) & \text{definição}.
    \end{align*}
\end{proof}

\hrule

\textbf{Exercício 8.8.} Para $n\ge1$ dê uma definição similar dos quantificadores "\textit{existe ao menos }$n$" e "\textit{existe exatamente }$n$".

\begin{proof}
    
\end{proof}

\hrule

\textbf{Exercício 8.9.} Sejam $P$ e $f$ binária e $x:=v_0,y:=v_1,u:=v_2,v:=v_3$ e $w:=v_4$. Mostre, usando a \textbf{Definição 8.2.} que:\\
a) $\exists xy(Pxu\wedge Pyv)\frac{u~u~u}{x~y~v}=\exists xy(Pxu\wedge Pyu)$;\\
b) $\exists xy(Pxu\wedge Pyv)\frac{v~fuv}{u~~~v}=\exists xy(Pxv\wedge Pyfuv)$;\\
c) $\exists xy(Pxu\wedge Pyv)\frac{u~x~fuv}{x~u~~~v}=\exists wy(Pwx\wedge Pyfuv)$;\\
d) $(\forall x\exists y(Pxy\wedge Pxu)\vee\exists ufuu\overtext{.}{=}x)\frac{x~fxy}{x~~~u}=\forall v\exists w(Pvw\wedge Pvfxy)\vee\exists ufuu\overtext{.}{=}x$.

\begin{proof}
    
\end{proof}

\hrule

\textbf{Exercício 8.10.} Mostre que se $x_0,\dots,x_r\notin\bigcup_{i=0}^r\msf{var}(t_i)$, então
$$\varphi\frac{t_0\dots t_r}{x_0\dots x_r}\vDash\Dashv\forall x_0\dots x_r(\bigwedge_{i=0}^rx_i\overtext{.}{=}t_i\rightarrow\varphi).$$

\begin{proof}
    
\end{proof}

\hrule

\textbf{Exercício 8.11.} Formalize um cálculo que derive strings exatamente da forma: $$tx_0\dots x_rt_0\dots t_rt\frac{t_0\dots t_r}{x_0\dots x_r}\text{ ou }\varphi x_0\dots x_rt_0\dots t_r\varphi\frac{t_0\dots t_r}{x_0\dots x_r}.$$

\begin{proof}
    
\end{proof}

\section{Cálculo de Sequentes}

\textbf{Exercício 2.7.} Analise quais das regras abaixo estão corretas:\\
\infer[  (i);]{\Gamma~(\varphi_1\vee\varphi_2)~(\psi_1\vee\psi_2)}{\Gamma~\varphi_1~\psi_1&&\Gamma~\varphi_2~\psi_2}~~~~~~
\infer[  (ii).]{\Gamma~(\varphi_1\vee\varphi_2)~(\psi_1\wedge\psi_2)}{\Gamma~\varphi_1~\psi_1&&\Gamma~\varphi_2~\psi_2}
\begin{proof}
    
\end{proof}

\hrule

\end{document}/