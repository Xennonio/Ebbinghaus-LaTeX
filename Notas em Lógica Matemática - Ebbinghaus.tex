\documentclass[11pt]{article}

%% Packages
\usepackage{amsmath,amsthm,amsfonts,amssymb,amscd}
\usepackage{multirow,booktabs}
\usepackage[table]{xcolor}
\usepackage{fullpage}
\usepackage{lastpage}
\usepackage{enumitem}
\usepackage{fancyhdr}
\usepackage{mathrsfs}
\usepackage{wrapfig}
\usepackage{setspace}
\usepackage{calc}
\usepackage{multicol}
\usepackage{cancel}
\usepackage[retainorgcmds]{IEEEtrantools}
\usepackage[margin=3cm]{geometry}
\usepackage{amsmath}
\usepackage{empheq}
\usepackage{framed}
\usepackage[most]{tcolorbox}
\usepackage{xcolor}
\usepackage{proof}
\usepackage{mathabx}

%% Pagestyle
\newlength{\tabcont}
\setlength{\parindent}{0.0in}
\setlength{\parskip}{0.05in}
\colorlet{shadecolor}{orange!15}
\parindent 0in
\parskip 12pt
\geometry{margin=1in, headsep=0.25in}
\theoremstyle{definition}
\newtheorem{defn}{Definição}
\newtheorem{exer}{Exercício}
\newtheorem{note}{Nota}
\newtheorem{theorem}{Teorema}
\newtheorem{corollary}{Corolário}
\newtheorem{lemma}{Lema}

%% NewCommands
\newcommand{\sse}{\leftrightarrow}
\newcommand{\mc}[1]{\mathcal{#1}}
\newcommand{\mf}[1]{\mathfrak{#1}}
\newcommand{\msf}[1]{\mathsf{#1}}
\newcommand{\mbb}[1]{\mathbb{#1}}
\newcommand{\ol}[1]{\overline{#1}}

%% Document

\begin{document}
\setcounter{section}{0}
\thispagestyle{empty}

\begin{center}
{\LARGE \bf Notas em Lógica Matemática}\\
{\large Ref. H. D. Ebbnghaus}\\
Primavera 2022
\end{center}

\tableofcontents

\section{Sintaxe de Linguagens de Primeira Ordem}

\subsection{Alfabetos}

\begin{shaded}
\begin{defn}
Um alfabeto $\mc{A}\ne\varnothing$ é um conjunto de símbolos. Denominamos uma sequência finita de símbolos em $\mc{A}$ strings ou palavras e denotamos por $\mc{A}^*$ o conjunto de todas elas. O comprimento ($\msf{len}:\mc{A}^*\to\mbb{N}$) de um $\zeta\in \mc{A}^*$ é o número de símbolos em $\mc{A}$ que ocorrem em $\zeta$. A string vazia $\square$ tq $\msf{len}(\square)=0$ também é considerada uma palavra.
\end{defn}
\end{shaded}

\begin{note}
\textbf{Mais a frente definiremos linguagens $L$ e é interessante ressaltar a distinção que tem de ser feita entre a linguagem objeto e a metalinguagem, a última é utilizada para se fazer a investigação sobre a primeira, que é o objeto de estudos, nessa formalização usaremos tanto a linguagem natural corrente quanto uma teoria dos conjuntos informalizada como metalinguagem, visto que dessa forma muitos conceitos já vistos anteriormentes podem ser reciclados.}
\end{note}

\begin{lemma}
Se $\mc{A}\preceq\aleph_0$ então $\mc{A}^*\approx\aleph_0$.
\end{lemma}

\subsection{O Alfabeto de uma Linguagem de Primeira Ordem}

\begin{shaded}
\begin{defn}
O alfabeto de uma linguagem de primeira ordem contém os símbolos:\\
(a) $v_0,v_1,v_2,\dots$ (variáveis);
(b) $\neg,\wedge,\vee,\to,\sse$ (não, e, ou, se-então, se e somente se (sse));\\
(c) $\forall,\exists$ (para todo, existe);\\
(d) $\equiv$ (igualdade);\\
(e) ),( (parênteses);\\
(f) um conjunto $\mc{S}=((1),(2),(3))$, possivelmente vazio, de assinaturas:\\
\phantom{aaa}(1) $\forall n\ge1$ um, possivelmente vazio, conjunto de símbolos de relações n-árias;\\
\phantom{aaa}(2) $\forall n\ge1$ um, possivelmente vazio, conjunto de símbolos de funções n-árias;\\
\phantom{aaa}(3) $\forall n\ge1$ um, possivelmente vazio, conjunto de constantes.
\end{defn}
\end{shaded}

$\mc{S}$ determina uma linguagem de primeira ordem $\mc{A}_\mc{S}:=\mc{A}\cup\mc{S}$

\begin{note}
\textbf{A partir de agora, usaremos $P,Q,R,\dots$ para símbolos de relações, $f,g,h,\dots$ para funções, $c,c_0,c_1,\dots$ para constantes e $x,y,z,\dots$ para variáveis.}
\end{note}

\subsection{Termos e Fórmulas em Linguagens de Primeira Ordem}

\begin{shaded}
\begin{defn}
Os $\mc{S}$-termos, elementos de $\mc{T}^\mc{S}$, são precisamente aquelas strings em $\mc{A}_\mc{S}^*$ obtidas por aplicações finitas das seguintes regras:\\
\text{(T1) Toda variável é um $\mc{S}$-termo};\\
\text{(T2) Toda constante é um $\mc{S}$-termo};\\
\text{(T3) Se $t_1,\dots,t_n$ é um $\mc{S}$-termo e $f\in\mc{S}$ um símbolo de função n-ária então $ft_1\dots t_n\in\mc{T}^\mc{S}$}.
\end{defn}
\end{shaded}

\begin{shaded}
\begin{defn}
As $\mc{S}$-fórmulas, elementos de $\mc{L}^\mc{S}$, são precisamente aquelas strings em $\mc{A}_\mc{S}^*$ obtidas por aplicações finitas das seguintes regras:\\
(F1) Se $t_1,t_2\in\mc{T}^\mc{S}$, então $t_1\equiv t_2$ é uma $\mc{S}$-fórmula;\\
(F2) Se $t_1,\dots,t_n\in\mc{T}^\mc{S}$ e $R\in\mc{S}$ é um símbolo de relação n-ária, então $Rt_1\dots t_n\in\mc{L}^\mc{S}$;\\
(F3) Se $\varphi\in\mc{L}^\mc{S}$, então $\neg\varphi\in\mc{L}^\mc{S}$;\\
(F4) Se $\varphi,\psi\in\mc{L}^\mc{S}$, então $(\varphi*\psi)\in\mc{L}^\mc{S}$, com $*=\wedge,\vee,\to,\sse$;\\
(F5) Se $\varphi\in\mc{L}^\mc{S}$ e $x$ é um variável, então $\forall x\varphi,\exists x\varphi\in\mc{L}^\mc{S}$.
\end{defn}
\end{shaded}

$\mc{S}$-fórmulas derivadas de (F1) e (F2) são ditas atômicas e as fórmulas $\neg\varphi,(\varphi*\psi)$ com $*=\wedge,\vee,\to,\sse$ são denominadas, respectivamente, negação de $\varphi$, conjunção, disjunção, implicação e bi-implicação.

\begin{note}
Por convenção não usaremos parênteses quando não houver ambiguidade e consideremos $\wedge,\vee$ associativos à esquerda, além de terem preferência em relação ao $\to$.
\end{note}

\begin{lemma}
Se $S\preceq\aleph_0$ então $\mc{T}^\mc{S},\mc{L}^\mc{S}\approx\aleph_0$.
\end{lemma}

\subsection{Indução no Cálculo de Termos e Fórmulas}
Seja $\mc{Z}\subset\mc{A}_\mc{S}^*$, quando $\mc{Z}=\mc{T}^\mc{S},\mc{L}^\mc{S}$ descrevemos uma lista de regra para sua construção que permitia a passagem de certas strings $\zeta_1,\dots,\zeta_n\in\mc{Z}$ para uma nova string $\zeta\in\mc{Z}$, podemos escrever isso esquematicamente da seguinte forma:
\[
\infer{\zeta}{\zeta_1,\dots,\zeta_n}
\]
Incluímos nesse esquema o caso "livre de premissas" que é quando $n=0$. Assim podemos escrever \textbf{Definition 3.} da seguinte forma:

\begin{align*}
    & \infer[(T1)]{x}{\phantom{aaaaaaa}};\\
    & \infer[(T2)]{c}{\phantom{aaaaaaa}}, \text{ se }c\in\mc{S};\\
    & \infer[(T3)]{ft_1\dots t_n}{t_1,\dots,t_n},\text{ se }f\in\mc{S}\text{ e }f\text{ é n-ária}.
\end{align*}

Quando definimos $\mc{Z}$ a partir de um cálculo $\mf{C}$ podemos fazer afirmações sobre os elementos de $\mc{Z}$ por meio de indução sobre $\mf{C}$. Para provar que todo elemento em $\mc{Z}$ tem uma propriedade $P$ é suficiente mostrar que todas as fórmulas livre de premissas deriváveis gozam de $P$ (hipótese de indução) e que toda regra em $\mf{C}$ preserva $P$. No caso particular em que $\mc{Z}=\mc{T}^\mc{S},\mc{L}^\mc{S}$ denominamos o procedimento de prova por indução em termos e fórmulas, respectivamente. Para provar que todo $\mc{S}$-termo goza de $P$ é suficiente mostrar:\\\\
(T1)' Toda variável goza de $P$;\\
(T2)' Toda constante em $\mc{S}$ goza de $P$;\\
(T3)' Se $t_1,\dots,t_n\in\mc{T}^\mc{S}$ goza de $P$ e $f\in\mc{S}$ é n-ária, então $ft_1\dots t_n$ também goza de $P$.\\\\
Para provar que toda $\mc{S}$-fórmula goza de $P$ é suficiente mostrar:\\\\
(F1)' Toda $\mc{S}$-fórmula da forma $t_1\equiv t_2$ goza de $P$;\\
(F2)' Toda $\mc{S}$-fórmula da forma $Rt_1\dots t_n$ goza de $P$;\\
(F3)' Se $\varphi\in\mc{L}^\mc{S}$ goza de $P$, então $\neg\varphi$ também;\\
(F4)' Se $\varphi,\psi\in\mc{L}^\mc{S}$ gozam de $P$, então $(\varphi*\psi)$, com $*=\wedge,\vee,\to,\sse$ também;\\
(F5)' Se $\varphi\in\mc{L}^\mc{S}$ goza de $P$ e $x$ é uma variável, então $\forall x\varphi,\exists x\varphi$ também.

\begin{lemma}
(a) $\forall t,t'\in\mc{T}^\mc{S}$, $t$ não é um segmento inicial próprio de $t'$ (i.e. $\neg\exists\zeta\ne\square$ tq $t\zeta=t'$);\\
(b)$\forall\varphi,\varphi'\in\mc{L}^\mc{S}$, $\varphi$ não é um segmento inicial próprio de $\varphi'$.
\end{lemma}

\begin{lemma}
(a) Se $t_1,\dots,t_n,t_1',\dots,t_m'\in\mc{T}^\mc{S}$ e $t_1\dots t_n=t_1'\dots t_m'$ então $m=n$ e $t_i=t_i',1\le i\le n$.
(b) Se $\varphi_1,\dots,\varphi_n,\varphi_1',\dots,\varphi_m'\in\mc{L}^\mc{S}$ e $\varphi_1\dots\varphi_n'=\varphi_1'\dots\varphi_m'$ então $m=n$ e $\varphi_i=\varphi_i',1\le i\le n$.
\end{lemma}

\begin{theorem}
Todo elemento de $\mc{T}^\mc{S}$ e $\mc{L}^\mc{S}$ é unicamente determinado pelos seus constituintes, i.e., possui uma única decomposição.
\end{theorem}

\begin{corollary}
É imediato que as condições abaixo são suficientes para definir uma função $f$ com $\msf{Dom}(f)=\mc{T}^\mc{S}$:\\
(T1)'' associar um valor a cada variável;\\
(T2)'' associar um valor a cada constante;\\
(T3)'' associar um valor a cada termo da forma $ft_1\dots t_n$ com $t_1,\dots,t_n$ já tendo valores associados.\\
Como tais fórmulas são unicamente determinadas a função existe.
\end{corollary}

\begin{shaded}
\begin{defn}
(a) A função $\msf{var}_\mc{S}$ (ou $\msf{var}$) associa a cada $\mc{S}$-termo o conjunto das variáveis que ocorrem nele:
\begin{align*}
    \msf{var}(x) & := {x}\\
    \msf{var}(c) & := \varnothing\\
    \msf{var}(ft_1\dots t_n) & := \bigcup_{n\in\mbb{N}}\msf{var}(t_n).
\end{align*}
(b) A função $\msf{SF}$, que associa a cada fórmula o conjunto das subfórmulas:
\begin{align*}
    \msf{SF}(t_1\equiv t_2) & := \{t_1\equiv t_2\}\\
    \msf{SF}(Rt_1\dots t_n) & := \{Rt_1\dots t_n\}\\
    \msf{SF}(\neg\varphi) & := \{\neg\varphi\}\cup\msf{SF}(\varphi)\\
    \msf{SF}((\varphi*\psi)) & := \{(\varphi*\psi)\}\cup\msf{SF}(\varphi)\cup\msf{SF}(\psi)\\
    & ~~~~ \text{ para }*=\wedge,\vee,\to,\sse\\
    \msf{SF}(\forall x\varphi) & := \{Q x\varphi\}\cup\msf{SF}(\varphi)\\
    & ~~~~ \text{ para }Q=\forall,\exists
\end{align*}
\end{defn}
\end{shaded}

\subsection{Variáveis Livres e Sentenças}

\begin{shaded}
\begin{defn}
A função $\msf{free}(\varphi)$ que associa a cada fórmula $\varphi$ o conjunto de variáveis livres nela:
\begin{align*}
    \msf{free}(t_1\equiv t_2) & := \msf{var}(t_1)\cup\msf{var}(t_2)\\
    \msf{free}(Pt_1\dots t_n) & := \bigcup_{n\in\mbb{N}}\msf{var}(t_n)\\
    \msf{free}(\neg\varphi) & := \msf{free}(\varphi)\\
    \msf{free}((\varphi*\psi)) & := \msf{free}(\varphi)\cup\msf{free}(\psi)\\
    & ~~~~*=\wedge,\vee,\to,\sse\\
    \msf{free}(Qx\varphi) & := \msf{free}(\varphi)\backslash\{x\}
\end{align*}
\end{defn}
\end{shaded}

Denotamos por $\mc{L}^\mc{S}_n:=\{\varphi\mid\varphi\in\mc{L}^\mc{S}\wedge\msf{free}(\varphi)\subset\{v_0,\dots,v_{n-1}\}\}$. Portanto o conjunto de $\mc{S}$-sentenças é denotado por $\mc{L}^\mc{S}_0$

\section{Semântica de Linguagens de Primeira Ordem}

\subsection{Estruturas e Interpretações}

\begin{shaded}
\begin{defn}
Uma $\mc{S}$-estrutura é um par $\mf{A}=(A,\mf{a})$ satisfazendo:\\
(a) $A\ne\varnothing$ é o domínio do discurso ou universo de $\mf{A}$, representado por $\msf{Dom}(\mf{A})$.\\
(b) $\mf{a}$ é uma mapeamento em $\mc{S}$ satisfazendo:\\
\phantom{aaa}(1) $\forall R\in\mc{S}$ símbolo de relação n-ária, $\mf{a}(R)\subseteq A^n$ é uma relação em $A$;\\
\phantom{aaa}(2) $\forall f\in\mc{S}$ símbolo de função n-ária, $\mf{a}(f):A^n\to A$;\\
\phantom{aaa}(3) $\forall c\in\mc{S}$ constante, $\mf{a}(c)\in A$.
\end{defn}
\end{shaded}

\begin{note}
\textbf{Denotaremos $\mf{a}(R),\mf{a}(f),\mf{a}(c)$ por $R^\mf{A},f^\mf{A},c^\mf{A}$, respectivamente, e uma ${R,f,g}$-estrura como sendo $\mf{A}=\left(A,R^\mf{a},f^\mf{a},g^\mf{a}\right)$. Quando a estrutura estiver subtendida escreveremos somente $\mf{A}=(A,R,f,g)$}
\end{note}

\begin{shaded}
\begin{defn}
Uma assinatura em uma $\mc{S}$-estrutura $\mf{A}$ é um mapeamento $\gamma:\{v_n\mid n\in\mbb{N}\}\to A$.
\end{defn}
\end{shaded}

\begin{shaded}
\begin{defn}
Uma $\mc{S}$-interpretação $\mf{I}$ é um par $(\mf{A},\gamma)$ consistindo de uma $\mc{S}$-estrutura $\mf{A}$ e uma assinatura $\gamma$ em $\mf{A}$.
\end{defn}
\end{shaded}

\begin{note}
\textbf{Se $\mu$ é uma assinatura em $\mf{B},a\in\msf{Dom}(\mf{B})$ e $x$ é uma variável, então $\mu\frac{a}{x}$ denota a assinatura que mapeia $x$ em $a$ e concorda com $\mu$ em todas as outras variáveis distintas de $x$:
\[
\mu\frac{a}{x}(y):=
\begin{cases}
    \mu(y) & y\ne x\\
    a & y=x
\end{cases}
\]
E para $\mf{I}=(\mf{B},\mu)$ temos $\mf{I}\frac{a}{x}:=\left(\mf{B},\mu\frac{a}{x}\right)$.
}
\end{note}

\subsection{Relação de Satisfação}

Definiremos o que $\mf{I}(t)$ significa, com $t\in\msf{Dom}(\mf{A})$ e $\mf{I}=(\mf{A},\beta)$, por indução nos termos:

\begin{shaded}
\begin{defn}
(a) Para uma variável $x$, $\mf{I}(x):=\beta(x)$;\\
(b) Para uma constante $c\in\mc{S}$, $\mf{I}(c):=c^\mf{A}$;\\
(c) Para um símbolo de função n-ária $f\in\mc{S}$ e $t_1,\dots,t_n\in\mc{T}^\mc{S}$:
\[
\mf{I}(ft_1\dots t_n):=f^\mf{A}(\mf{I}(t_1),\dots,\mf{I}(t_n)).
\]
\end{defn}
\end{shaded}

Agora definiremos a relação de satisfação.

\begin{shaded}
\begin{defn}
Para todo $\mf{I}=(\mf{A},\beta)$ temos:
\begin{align*}
    \mf{I}\vDash t_1\equiv t_2 & ~~~\text{sse}~~~ \mf{I}(t_1)=\mf{I}(t_2);\\
    \mf{I}\vDash Rt_1\dots t_n & ~~~\text{sse}~~~ R^\mf{A}\mf{I}(t_1)\dots\mf{I}(t_n);\\
    \mf{I}\vDash\neg\varphi & ~~~\text{sse}~~~\text{não ocorre }\mf{I}\vDash\varphi;\\
    \mf{I}\vDash(\varphi\wedge\psi) & ~~~\text{sse}~~~ \mf{I}\vDash\varphi\text{ e }\mf{I}\vDash\psi;\\
    \mf{I}\vDash(\varphi\vee\psi) & ~~~\text{sse}~~~ \mf{I}\vDash\varphi\text{ ou }\mf{I}\vDash\psi;\\
    \mf{I}\vDash(\varphi\to\psi) & ~~~\text{sse}~~~ \text{se } \mf{I}\vDash\varphi\text{ então }\mf{I}\vDash\psi;\\
    \mf{I}\vDash(\varphi\sse\psi) & ~~~\text{sse}~~~ \mf{I}\vDash\varphi\text{ sse }\mf{I}\vDash\psi;\\
    \mf{I}\vDash\forall x\varphi & ~~~\text{sse}~~~\text{para todo }a\in A,\mf{I}\frac{a}{x}\vDash\varphi;\\
    \mf{I}\vDash\exists x\varphi & ~~~\text{sse}~~~\text{existe um }a\in A\text{ tq }\mf{I}\frac{a}{x}\vDash\varphi.
\end{align*}
\end{defn}
\end{shaded}

\begin{note}
\textbf{Dado um conjunto $\Phi$ de $\mc{S}$-fórmulas, dizemos que $\mf{I}$ é um modelo de $\Phi$ e escrevemos $\mf{I}\vDash\Phi$ se $\mf{I}\vDash\varphi$ para todo $\varphi\in\Phi$}
\end{note}

Seja $\msf{Pos}^\mc{S}$ o conjunto de $S$-fórmulas positivas, definimos $\msf{Pos}^\mc{S}$ indutivamente:
\[
\infer[\varphi\text{ é atômica};]{\varphi}{\phantom{aaaaa}}~~~~\infer[*=\wedge,\vee,\to,\sse;]{(\varphi*\psi)}{\varphi & \psi}~~~~\infer[x\text{ é uma variável},Q=\forall,\exists.]{Qx\varphi}{x & \varphi}
\]
Seja $\mf{I}=(\mf{A},\beta)$ uma $\mc{S}$-interpretação onde $\mf{A}=(c^\mf{A},\mf{a})$ é uma $\mc{S}$-estrutura. Então $\forall c\in\mc{S}\left(\mf{a}(c)=c^\mf{A}\right)$, $\forall v\left(\beta(v)=c^\mf{A}\right)$, sendo $v$ uma variável, e $\forall c\left(\mf{I}(c)=c^\mf{A}\right)$. Seja $\mc{P}(\varphi):=\mf{I}\vDash\varphi$ provaremos por indução em fórmulas que $\mc{P}$ vale para todo elemento em $\msf{Pos}^\mc{S}$:\\
Hipótese de indução: $\mc{P}(\varphi)$ onde $\varphi$ é atômica:\\
$\varphi=t_1\equiv t_2$: $\mf{I}\vDash\varphi$ sse $\mf{I}(t_1)=c^\mf{A}=\mf{I}(t_2)$.\\
$\varphi=Rt_1\dots t_n$: como não temos $R$ na estrutura então é satisfeito por vacuidade.\\
$\varphi=(\psi*\chi)$: como pela hipótese de indução $\mf{I}\vDash\psi$ e $\mf{I}\vDash\chi$ então $\mf{I}\vDash\varphi$\\
$\varphi=\forall x\psi$: $\mc{P}(\varphi)$ sse para todo $a\in A$, i.e. $c^\mf{A}$, $\mf{I}\frac{a}{x}\vDash\psi$ mas $\mf{I}\frac{a}{x}=(\mf{A},\beta\frac{a}{x})=(\mf{A},\beta)=\mf{I}$ portanto, pela hipótese de indução, $\mf{I}\vDash\psi$ então $\mf{I}\vDash\varphi$.\\
$\varphi=\exists x\varphi$: O argumento é análogo ao anterior.

\subsection{A Relação de Consequência}

\begin{shaded}
\begin{defn}
Seja $\Phi$ um conjunto de fórmulas e $\varphi$ uma fórmula. $\varphi$ é uma consequência de $\Phi$ (escrito $\Phi\vDash\varphi$) sse toda interpretação que é um modelo de $\Phi$ também é modelo de $\varphi$.
\end{defn}
\end{shaded}

\begin{note}
\textbf{Se $\Phi=\{\psi\}$ escrevemos $\psi\vDash\varphi$ ao invés de $\{\psi\}\vDash\varphi$.}
\end{note}

\begin{shaded}
\begin{defn}
Uma fórmula $\varphi$ é válida (escrito $\vDash\varphi$) sse $\varnothing\vDash\varphi$.
\end{defn}
\end{shaded}

Assim uma fórmula é válida sse toda interpretação é um modelo dela.

\begin{shaded}
\begin{defn}
Uma fórmula $\varphi$ é satisfatível (escrito $\msf{Sat}(\varphi)$) sse há uma interpretação que é um modelo de $\varphi$. Para um conjunto de fórmulas $\Phi$ este é satisfatível ($\msf{Sat}(\Phi)$) sse existe uma interpretação que é modelo de todas as fórmulas em $\Phi$.
\end{defn}
\end{shaded}

\begin{lemma}
Para todo $\Phi$ e $\varphi$
\[
\Phi\vDash\varphi\text{ sse não é o caso que }\msf{Sat}(\Phi\cup\{\neg\varphi\}).
\]
Em particular, $\varphi$ é válida sse $\neg\varphi$ não é satisfatível.
\end{lemma}

\begin{shaded}
\begin{defn}
Duas fórmulas $\varphi$ e $\psi$ são logicamente equivalentes (escrito $\varphi\vDash\Dashv\psi$) sse $\varphi\vDash\psi$ e $\psi\vDash\varphi$. Portanto $\varphi\vDash\Dashv\psi$ sse ambas são válidas nas mesmas interpretações i.e. $\vDash\varphi\sse\psi$.
\end{defn}
\end{shaded}

Evidentemente as seguintes fórmulas são equivalentes:
\begin{align*}
    \varphi\wedge\psi &\vDash\Dashv \neg(\neg\varphi\vee\neg\psi)\\
    \varphi\to\psi &\vDash\Dashv \neg\varphi\vee\psi\\
    \varphi\sse\psi &\vDash\Dashv \neg(\varphi\vee\psi)\vee\neg(\neg\varphi\vee\neg\psi)\\
    \forall x\varphi &\vDash\Dashv \neg\exists x\neg\varphi.
\end{align*}
Portanto é possível definir um mapeamento $*$ por indução em fórmulas que associa a cada $\varphi$ um $\varphi*$ tal que $\varphi\vDash\Dashv\varphi*$ e não contém $\wedge,\to,\sse,\forall$ o que diminui as provas por indução em fórmulas.
O lema a seguir expressa a formulação exata do - intuitivamente claro - fato que a relação de satisfatibilidade entre uma $\mc{S}$-interpretação $\mf{I}$ e uma $\mc{S}$-fórmula $\varphi$ depende somente da interpretação dos $\mc{S}$-símbolos e das variáveis livres que ocorrem em $\varphi$.

\begin{lemma}
\textbf{Lema da Coincidência.} Seja $\mf{I}_1=(\mf{A}_1,\beta_1)$ uma $\mc{S}_1$-interpretação e $\mf{I}_2=(\mf{A}_2,\beta_2)$ uma $\mc{S}_2$-interpretação tq $\msf{Dom}(\mf{A}_1)=\msf{Dom}(\mf{A}_2)$, seja $\mc{S}:=\mc{S}_1\cap\mc{S}_2$:\\
(a) Seja $t$ um $\mc{S}$-termo. Se $\mf{I}_1$ e $\mf{I}_2$ concordam nos $\mc{S}$-símbolos, i.e. $\kappa^\mf{A_1}=\kappa^\mf{A_2}$, e variáveis, i.e. $\beta_1(x)=\beta_2(x)$, que ocorrem em $t$, então $\mf{I}_1(t)=\mf{I}_2(t)$;\\
(b) Seja $\varphi$ uma $\mc{S}$-fórmula. Se $\mf{I}_1$ e $\mf{I}_2$ concordam nos $\mc{S}$-símbolos e nas variáveis que ocorrem livre em $\varphi$, então $\mf{I}_1\vDash\varphi$ e $\mf{I}_2\vDash\varphi$.
\end{lemma}

Se $\varphi\in\mc{L}^\mc{S}_n$ pelo teorema acima somente os valores $a_i=\beta(v_i),i=0,\dots,n-1$ são significantes, portanto introduzimos a seguinte notação:

\begin{note}
\textbf{Ao invés de $(\mf{A},\beta)\vDash\varphi$ escreveremos:
\[
\mf{A}\vDash\varphi[a_0,\dots,a_{n-1}].
\]
Da mesma forma, se $\msf{var}(t)\subset\{v_0,\dots,v_{n-1}\}$ então ao invés de $\mf{I}(t)$ escrevemos $t^\mf{A}[a_0,\dots,a_{n-1}]$.}
\end{note}

Se $n=0$ escrevemos $\mf{A}\vDash\varphi$ e dizemos que $\mf{A}$ é um modelo de $\varphi$ ou, para um conjunto de sentenças $\Phi$, $\mf{A}\vDash\Phi$ significa que $\mf{A}\vDash\varphi$ para todo $\varphi\in\Phi$.

\begin{shaded}
\begin{defn}
Sejam $\mc{S}\subset\mc{S}'$ conjuntos de símbolos e $\mf{A}=(A,\mf{a}),\mf{A}'=(A',\mf{a}')$ $\mc{S}$ e $\mc{S}'$ estruturas, respectivamente. Dizemos que $\mf{A}$ é uma $\mc{S}$-redução de $\mf{A}'$ (ou que $\mf{A}'$ é uma $\mc{S}$ expansão de $\mf{A}$) sse $A=A'$ e $\mf{a},\mf{a}'$ concordam em $\mc{S}$. Denotamos por $\mf{A}=\mf{A}'\mid_\mc{S}$.
\end{defn}
\end{shaded}

Note que a definição de interpretação, satisfatilibidade e consequência se referem a um conjunto de símbolos $\mc{S}$ fixo. Entretanto é possível remover tal referência a partir do \textbf{Lema da Coincidência}.

\begin{corollary}
$\Phi$ é satisfatível com respeito a $\mc{S}$ sse também é com respeito a $\mc{S}'$.
\end{corollary}

\subsection{Dois Lemas Sobre a Relação de Satisfatibilidade}

Os resultados a seguir serão sobre estruturas e subestruturas isomórficas.

\begin{shaded}
\begin{defn}
Sejam $\mf{A}$ e $\mf{B}$ $\mc{S}$-estruturas.\\
(a) Um mapeamento $\pi:\msf{Dom}(\mf{A})\to\msf{Dom}(\mf{B})$ é denominado um \textit{isomorfismo} de $\mf{A}$ em $\mf{B}$ (denotado $\pi:\mf{A}\cong\mf{B}$) sse\\
(i) $\pi$ é uma bijeção de $\msf{Dom}(\mf{A})$ em $\msf{Dom}(\mf{B})$;\\
(ii) Para $R,f\in\mc{S}$ n-árias, $c\in\mc{S}$ e $a_1,\dots,a_n\in\msf{Dom}(\mf{A})$:
\[
R^\mf{A}a_1\dots a_n\text{ sse }R^\mf{B}\pi(a_1)\dots\pi(a_n);
\]
\[
\pi(f^\mf{A}(a_1,\dots,a_n))=f^\mf{B}(\pi(a_1),\dots,\pi(a_n));
\]
\[
\pi(c^\mf{A})=c^\mf{B}.
\]
(b) Estruturas $\mf{A}$ e $\mf{B}$ são ditas \textit{isomórficas} (denotado $\mf{A}\cong\mf{B}$) sse há um isomorfismo $\pi:\mf{A}\cong\mf{B}$. 
\end{defn}
\end{shaded}

O lema a seguir mostra que sentenças de primeira ordem não conseguem distinguir estruturas isomórficas:

\begin{lemma}
\textbf{Lema do Isomorfismo.} Para $\mc{S}$-estruturas isomórficas $\mf{A}$ e $\mf{B}$ e toda $\mc{S}$-sentença $\varphi$:
\[
\mf{A}\vDash\varphi\text{ sse }\mf{B}\vDash\varphi.
\]
\end{lemma}

\begin{corollary}
Se $\pi:\mf{A}\cong\mf{B}$ então para $\varphi\in\mc{L}^\mc{S}_n$ e $a_0,\dots,a_{n-1}\in\msf{Dom}(\mf{A})$:
\[
\mf{A}\vDash\varphi[a_0,\dots,a_{n-1}]\text{ sse }\mf{B}\vDash\varphi[a_0,\dots,a_{n-1}]
\]
Estruturas isomórficas são indistinguíveis em $\mc{L}^\mc{S}_0$
\end{corollary}

\begin{shaded}
\begin{defn}
Sejam $\mf{A}$ e $\mf{B}$ $\mc{S}$-estruturas. Então $\mf{A}$ é dito subestrutura de $\mf{B}$ (denotado $\mf{A}\subseteq\mf{B}$) sse\\
(a) $\msf{Dom}(\mf{A})\subseteq\msf{Dom}(\mf{B})$\\
(b)~~(1) Para $R\in\mc{S}$ n-ário, $R^\mf{A}=R^\mf{B}\cap\msf{Dom}(\mf{A})^n$\\
\phantom{aaaa}(2) Para $f\in\mc{S}$ n-ário, $f^\mf{A}=f^\mf{B}\mid_{\msf{Dom}(\mf{A})^n}$\\
\phantom{aaaa}(3) Para $c\in\mc{S}$ $c^\mf{A}=c^\mf{B}$.
\end{defn}
\end{shaded}

\begin{lemma}
Sejam $\mf{A}$ e $\mf{B}$ $\mc{S}$-estruturas tq $\mf{A}\subseteq\mf{B}$ e seja $\beta:\{v_n\mid n\in\mbb{N}\}\to\msf{Dom}(\mf{A})$ uma assinatura em $\mf{A}$. Então para todo $\mc{S}$-termo $t$ vale:
\[
(\mf{A},\beta)(t)=(\mf{B},\beta)(t);
\]
E para toda $\mc{S}$-fórmulas livre de quantificadores $\varphi$:
\[
(\mf{A},\beta)\vDash\varphi\text{ sse }(\mf{B},\beta)\vDash\varphi.
\]
\end{lemma}

\begin{shaded}
\begin{defn}
As fórmulas deriváveis no seguinte cálculo são ditas \textit{fórmulas universais}:
\begin{align*}
    \text{(i) }\infer[\text{Se }\varphi\text{ é livre de quantificadores}]{\varphi}{\phantom{aaa}} & ~~~~~~~~  \text{(ii) } \infer[*=\wedge,\vee;]{(\varphi*\psi)}{\varphi & \psi}\\
    \text{(iii) }\infer[.]{\forall x\varphi}{\varphi}
\end{align*}
\end{defn}
\end{shaded}

\begin{note}
\textbf{Todas as fórmulas universais são logicamente equivalentes as da forma $\forall x_1\dots\forall x_n\psi$ sendo $\psi$ livre de quantificadores.}
\end{note}

\begin{lemma}
\textbf{Lema da Subestrutura.} Sejam $\mf{A}$ e $\mf{B}$ $\mc{S}$-estruturas tq $\mf{A}\subseteq\mf{B}$ e $\varphi\in\mc{L}^\mc{S}_n$ uma fórmula universal. Então para todo $a_0,\dots,a_{n-1}\in\msf{Dom}(\mf{A})$:
\[
\text{Se }\mf{B}\vDash\varphi[a_0,\dots,a_{n-1}]\text{, então }\mf{A}\vDash\varphi[a_0,\dots,a_{n-1}].
\]
\end{lemma}

\begin{corollary}
Se $\mf{A}\subseteq\mf{B}$ então para toda sentença universal $\varphi$:
\[
\text{Se }\mf{B}\vDash\varphi\text{, então }\mf{A}\vDash\varphi.
\]
\end{corollary}

\subsection{Substituição}

\begin{shaded}
\begin{defn}
\begin{align*}
x\frac{t_0\dots t_r}{x_0\dots x_r} & :=
\begin{cases}
x & \text{ se }x\ne x_0,\dots,x\ne x_r\\
t_i & \text{ se }x=x_i
\end{cases}\\
c\frac{t_0\dots t_r}{x_0\dots x_r} & :=c\\
[ft'_1\dots t'_n]\frac{t_0\dots t_r}{x_0\dots x_r} & :=ft'_1\frac{t_0\dots t_r}{x_0\dots x_r}\dots t'_n\frac{t_0\dots t_r}{x_0\dots x_r}\\
[t'_1\equiv t'_2]\frac{t_0\dots t_r}{x_0\dots x_r} & :=t'_1\frac{t_0\dots t_r}{x_0\dots x_r}\equiv t'_2\frac{t_0\dots t_r}{x_0\dots x_r}\\
[Rt'_1\dots t'_n]\frac{t_0\dots t_r}{x_0\dots x_r} & :=Rt'_1\frac{t_0\dots t_r}{x_0\dots x_r}\dots t'_n\frac{t_0\dots t_r}{x_0\dots x_r}\\
[\neg\varphi]\frac{t_0\dots t_r}{x_0\dots x_r} & :=\neg[\varphi\frac{t_0\dots t_r}{x_0\dots x_r}]\\
(\varphi\vee\psi)\frac{t_0\dots t_r}{x_0\dots x_r} & :=\left(\varphi\frac{t_0\dots t_r}{x_0\dots x_r}\vee\psi\frac{t_0\dots t_r}{x_0\dots x_r}\right)
\end{align*}

(h) Sejam $x_{i_1},\dots,x_{i_s} (i_1<\dots<i_s)$ as variáveis $x_i$ entre $x_0,\dots,x_r$ tq
\[
x_i\in\msf{free}(\exists x\varphi), x_i\ne t_i
\]
com $x\ne x_{i_1},\dots,x\ne x_{i_s}$, então
\[
[\exists x\varphi]\frac{t_0\dots t_r}{x_0\dots x_r}:=\exists u\left[\varphi\frac{t_0\dots t_r}{x_0\dots x_r}\right]
\]
\end{defn}
\end{shaded}

Para a generalização da definição de $\mf{I}\frac{a}{x}$ temos:

\begin{shaded}
\begin{defn}
Sejam $x_0,\dots,x_r$ distintos dois a dois e seja $\mf{I}=(\mf{A},\beta)$ com $a_0,\dots,a_r\in\msf{Dom}(\mf{A})$:
\[
\beta\frac{a_0\dots a_r}{x_0\dots x_r}:=\begin{cases}
\beta(y) & \text{ se }y\ne x_0,\dots,y\ne x_r\\
a_i & \text{ se }y=x_i
\end{cases}
\]
\[
\mf{I}\frac{a_0\dots a_r}{x_0\dots x_r}:=\left(\mf{A},\beta\frac{a_0\dots a_r}{x_0\dots x_r}\right).
\]
\end{defn}
\end{shaded}

\begin{lemma}
\textbf{Lema da Substituição.} (a) Para todo termo $t$:
\[
\mf{I}\left(t\frac{t_0\dots t_r}{x_0\dots x_r}\right)=\mf{I}\frac{\mf{I}(t_0)\dots\mf{I}(t_r)}{x_0\dots x_r}(t).
\]
(b) Para toda fórmula $\varphi$:
\[
\mf{I}\vDash\varphi\frac{t_0\dots t_r}{x_0\dots x_r}\text{ sse }\mf{I}\frac{\mf{I}(t_0)\dots\mf{I}(t_r)}{x_0\dots x_r}\vDash\varphi.
\]
\end{lemma}

\begin{lemma}
Para toda permutação $\pi$ de $\{0,\dots,r\}$:\\
(a)
\[
\varphi\frac{t_0\dots t_r}{x_0\dots x_r}=\varphi\frac{t_{\pi(0)}\dots t_{\pi(r)}}{x_{\pi(0)}\dots x_{\pi(r)}}.
\]
(b) Se $0\le i\le r$ e $x_i=t_i$, então
\[
\varphi\frac{t_0\dots t_r}{x_0\dots x_r}=\varphi\frac{t_0\dots t_{i-1}~t_{i+1}\dots t_r}{x_0\dots x_{i-1}~x_{i+1}\dots x_r}.
\]
(c) Para toda variável $y$\\
(i) Se $y\in\msf{var}\left(t\frac{t_0\dots t_r}{x_0\dots x_r}\right)$, então $y\in\bigcup_{0\le i\le r}\msf{var(t_i)}$ ou ($y\in\msf{var}(t)$ e $y\ne x_0,\dots,y\ne x_r$);\\
(ii) Se $y\in\msf{var}\left(\varphi\frac{t_0\dots t_r}{x_0\dots x_r}\right)$, então $y\in\bigcup_{0\le i\le r}\msf{var(t_i)}$ ou ($y\in\msf{var}(\varphi)$ e $y\ne x_0,\dots,y\ne x_r$).\\
\end{lemma}

\begin{corollary}
Suponha $\msf{free}(\varphi)\subseteq\{x_0,\dots,x_r\}$ distintos dois a dois.
\end{corollary}

\end{document}